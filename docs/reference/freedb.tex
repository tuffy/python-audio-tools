\chapter{FreeDB}
Because compact discs do not usually contain metadata about
track names, album names and so forth, that information
must be retrieved from an external source.
FreeDB is a service which allows users to submit CD
metadata and to retrieve the metadata submitted by others.
Both actions require a category and a 32-bit disc ID number,
which combine to form a unique identifier for a particular CD.

%% The 32-bit disc ID is calculated from the total number of tracks,
%% the track offsets (in CD sectors) and the total length of the CD
%% (in seconds).

%% In the case of CD submission, the genre is decided by the submitter.
%% In the case of CD retrieval, the user must choose from a list
%% of possible choices when there is a collision between 32-bit disc ID
%% numbers.

%% The actual metadata is stored as XMCD files.

\begin{wrapfigure}[18]{r}{1.24in}
\includegraphics{figures/freedb_sequence.pdf}
\end{wrapfigure}
\section{Native Protocol}
FreeDB's native protocol runs as a service on TCP port 8880.
\begin{itemize}
\item After connecting, the client and server exchange a handshake using
the \texttt{hello} command.
The server will not do anything without this handshake.

\item Next the client changes to protocol level 6 with
the \texttt{proto} command.
This is necessary because only the highest protocol supports
UTF-8 text encoding.
Without this, any characters not in the latin-1 set will not be
sent properly.

\item Once that is accomplished, the client should calculate the 32-bit
disc ID from the track information.

\item One then sends the 32-bit disc ID and additional disc information
to the server with the \texttt{query} command to retrieve a list of
matching disc IDs, genres and titles.
If there are multiple matches, the user must be prompted to
choose one of the matches.

\item When our match is known, the client uses the \texttt{read} command
to retrieve the actual XMCD data.

\item Finally, the \texttt{close} command is used to sever the connection
and complete the transaction.
\end{itemize}

\pagebreak

\section{the disc ID}
FreeDB uses a big-endian 32-bit disc ID to differentiate
on disc from another.

\begin{figure}[h]
\includegraphics{figures/freedb_discid.pdf}
\end{figure}
\par
\noindent
`Track Count' is self-explanatory.
`Total Length' is the total length of all the tracks, not
counting the initial 2 second lead-in.
`Offset Seconds Digit Sum' is the sum of the digits of all
the disc's track offsets, in seconds, and truncated to 8 bits.
Remember to count the initial 2 second/150 frame lead-in when calculating
offsets.
\begin{table}[h]
\begin{tabular}{|c||r|r|r||r|r|r|}
\hline
Track & \multicolumn{3}{c||}{Length} & \multicolumn{3}{c|}{Offset} \\
Number & in M:SS & in seconds & in frames & in M:SS & in seconds & in frames \\
\hline
1 & 3:37 & 217 & 16340 & 0:02 & 2 & 150 \\
2 & 3:23 & 203 & 15294 & 3:39 & 219 & 16490 \\
3 & 3:37 & 217 & 16340 & 7:03 & 423 & 31784 \\
4 & 3:20 & 200 & 15045 & 10:41 & 641 & 48124 \\
\hline
\end{tabular}
\end{table}
\par
\noindent
In this example, `Track Count' is \textbf{4}.
`Total Length' is
$\frac{16340 + 15294 + 16340 + 15045}{75} = \textbf{840}$

There are 75 frames per second, and one must remember to count
fractions of seconds when calculating the total disc length.

The `Offset Seconds Digit Sum' is calculated by looking at the
`Offset in Seconds' column.
Those values are 2, 219, 423 and 641.
One must take all of those digits and add them, which works out to
$2 + 2 + 1 + 9 + 4 + 2 + 3 + 6 + 4 + 1 = \textbf{34}$

This means our three values are 34, 840 and 4.
In hexadecimal, they are 0x22, 0x0348 and 0x04.
Combining them into a single value yields 0x22034804.
Thus, our FreeDB disc ID is \texttt{22034804}
