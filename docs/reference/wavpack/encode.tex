%Copyright (C) 2007-2014  Brian Langenberger
%This work is licensed under the
%Creative Commons Attribution-Share Alike 3.0 United States License.
%To view a copy of this license, visit
%http://creativecommons.org/licenses/by-sa/3.0/us/ or send a letter to
%Creative Commons,
%171 Second Street, Suite 300,
%San Francisco, California, 94105, USA.

\section{WavPack Encoding}

{\relsize{-1}
  \input{wavpack/algorithms/encode_wavpack}
}

\clearpage

\subsection{Determine Block Split}
\label{wavpack:block_split}
\ALGORITHM{input stream's channel assignment}{number of blocks per set, list of channel counts per block}
\SetKwData{BLOCKCOUNT}{block count}
\SetKwData{BLOCKCHANNELS}{block channels}
\Switch(\tcc*[f]{split channels by left/right pairs}){channel assignment}{
  \uCase{mono}{
    $\text{\BLOCKCOUNT} \leftarrow 1$\;
    $\text{\BLOCKCHANNELS} \leftarrow \texttt{[1]}$\;
  }
  \uCase{front left, front right}{
    $\text{\BLOCKCOUNT} \leftarrow 1$\;
    $\text{\BLOCKCHANNELS} \leftarrow \texttt{[2]}$\;
  }
  \uCase{front left, front right, front center}{
    $\text{\BLOCKCOUNT} \leftarrow 2$\;
    $\text{\BLOCKCHANNELS} \leftarrow \texttt{[2, 1]}$\;
  }
  \uCase{front left, front right, back left, back right}{
    $\text{\BLOCKCOUNT} \leftarrow 2$\;
    $\text{\BLOCKCHANNELS} \leftarrow \texttt{[2, 2]}$\;
  }
  \uCase{front left, front right, front center, back center}{
    $\text{\BLOCKCOUNT} \leftarrow 3$\;
    $\text{\BLOCKCHANNELS} \leftarrow \texttt{[2, 1, 1]}$\;
  }
  \uCase{front left, front right, front center, back left, back right}{
    $\text{\BLOCKCOUNT} \leftarrow 3$\;
    $\text{\BLOCKCHANNELS} \leftarrow \texttt{[2, 1, 2]}$\;
  }
  \uCase{front left, front right, front center, LFE, back left, back right}{
    $\text{\BLOCKCOUNT} \leftarrow 4$\;
    $\text{\BLOCKCHANNELS} \leftarrow \texttt{[2, 1, 1, 2]}$\;
  }
  \Other(\tcc*[f]{save them independently}){
    $\text{\BLOCKCOUNT} \leftarrow$ channel count\;
    $\text{\BLOCKCHANNELS} \leftarrow$ 1 per channel\;
  }
}
\Return \BLOCKCOUNT and \BLOCKCHANNELS
\EALGORITHM
\vskip 1ex
\par
\noindent
One could invent alternate channel splits for other obscure assignments.
WavPack's only requirement is that all channels must be in
Wave order\footnote{see page \pageref{wave_channel_assignment}}
and each block must contain 1 or 2 channels.

\begin{figure}[h]
\includegraphics{wavpack/figures/block_channels.pdf}
\end{figure}

\clearpage

\subsection{Encoding Parameters}
\label{wavpack:encoding_parameters}
{\relsize{-1}
  \input{wavpack/algorithms/encoding_parameters}
}

\clearpage

\subsection{Encoding Block}
\label{wavpack:encode_block}
{\relsize{-2}
  \input{wavpack/algorithms/encode_block}
}

\clearpage

\subsection{Calculating Maximum Magnitude}
\label{wavpack:calc_maximum_magnitude}
{\relsize{-1}
  \input{wavpack/algorithms/calculate_max_magnitude}
}

\subsection{Calculating Wasted Bits Per Sample}
\label{wavpack:calc_wasted_bps}
{\relsize{-1}
  \input{wavpack/algorithms/calculate_wasted_bps}
where the \texttt{wasted} function is defined as:
\begin{equation*}
\texttt{wasted}(x) =
\begin{cases}
\infty & \text{if } x = 0 \\
0 & \text{if } x \bmod 2 = 1 \\
1 + \texttt{wasted}(x \div 2) & \text{if } x \bmod 2 = 0 \\
\end{cases}
\end{equation*}
}

\clearpage

\subsection{Joint Stereo Conversion}
\label{wavpack:calc_joint_stereo}
\input{wavpack/algorithms/apply_joint_stereo}

\subsubsection{Joint Stereo Example}
\begin{table}[h]
{\relsize{-1}
\begin{tabular}{|r|r|r||>{$}r<{$}|>{$}r<{$}|}
$i$ & $\textsf{left}_i$ & $\textsf{right}_i$ & \textsf{mid}_i & \textsf{side}_i \\
\hline
0 & 0 & 64 & 0 - 64 = -64 & \lfloor(0 + 64) \div 2\rfloor = 32 \\
1 & 16 & 62 & 16 - 62 = -46 & \lfloor(16 + 62) \div 2\rfloor = 39 \\
2 & 31 & 56 & 31 - 56 = -25 & \lfloor(31 + 56) \div 2\rfloor = 43 \\
3 & 44 & 47 & 44 - 47 = -3 & \lfloor(44 + 47) \div 2\rfloor = 45 \\
4 & 54 & 34 & 54 - 34 = 20 & \lfloor(54 + 34) \div 2\rfloor = 44 \\
5 & 61 & 20 & 61 - 20 = 41 & \lfloor(61 + 20) \div 2\rfloor = 40 \\
6 & 64 & 4 & 64 - 4 = 60 & \lfloor(64 + 4) \div 2\rfloor = 34 \\
7 & 63 & -12 & 63 - -12 = 75 & \lfloor(63 + -12) \div 2\rfloor = 25 \\
8 & 58 & -27 & 58 - -27 = 85 & \lfloor(58 + -27) \div 2\rfloor = 15 \\
9 & 49 & -41 & 49 - -41 = 90 & \lfloor(49 + -41) \div 2\rfloor = 4 \\
\end{tabular}
}
\end{table}

\clearpage

\subsection{Writing Block Parameters}
\label{wavpack:write_block_parameters}
{\relsize{-1}
  \input{wavpack/algorithms/write_block_parameters}
}

\clearpage

\subsection{Writing Sub Block Header}
\label{wavpack:write_sub_block_header}
\input{wavpack/algorithms/write_sub_block_header}

\clearpage

%This work is licensed under the
%Creative Commons Attribution-Share Alike 3.0 United States License.
%To view a copy of this license, visit
%http://creativecommons.org/licenses/by-sa/3.0/us/ or send a letter to
%Creative Commons,
%171 Second Street, Suite 300,
%San Francisco, California, 94105, USA.

\subsection{Writing Decorrelation Terms}
\label{wavpack:write_decorr_terms}
\ALGORITHM{a list of decorrelation terms, a list of decorrelation deltas}{decorrelation terms sub block data}
\SetKwData{TERM}{term}
\SetKwData{DELTA}{delta}
\SetKwData{KwDownTo}{downto}
\For(\tcc*[f]{populate in reverse order}){$p \leftarrow \text{decorrelation pass count}$ \emph{\KwDownTo}0}{
  $\text{\TERM}_p + 5 \rightarrow$ \WRITE 5 unsigned bits\;
  $\text{\DELTA}_p \rightarrow$ \WRITE 3 unsigned bits\;
}
\Return decorrelation terms sub block data\;
\EALGORITHM
\begin{figure}[h]
  \includegraphics{wavpack/figures/decorr_terms.pdf}
\end{figure}

\clearpage
For example, given decorrelation terms and deltas:
\begin{table}[h]
\begin{tabular}{rrr}
$p$ & $\textsf{term}_p$ & $\textsf{delta}_p$ \\
\hline
0 & 3 & 2 \\
1 & 17 & 2 \\
2 & 2 & 2 \\
3 & 18 & 2 \\
4 & 18 & 2 \\
\end{tabular}
\end{table}
\par
\noindent
the decorrelation terms sub block is written as:
\begin{figure}[h]
\includegraphics{wavpack/figures/terms_parse.pdf}
\end{figure}


\clearpage

%This work is licensed under the
%Creative Commons Attribution-Share Alike 3.0 United States License.
%To view a copy of this license, visit
%http://creativecommons.org/licenses/by-sa/3.0/us/ or send a letter to
%Creative Commons,
%171 Second Street, Suite 300,
%San Francisco, California, 94105, USA.

\subsection{Decode Decorrelation Weights}
\label{wavpack:decode_decorrelation_weights}
{\relsize{-1}
\ALGORITHM{\VAR{mono output} and \VAR{false stereo} from block header, decorrelation terms count\footnote{from the decorrelation terms sub block, which must be read prior to this sub block}, \VAR{actual size 1 less} and \VAR{sub block size} values from sub block header, sub block data}{a list of signed weight integers per channel\footnote{$\text{weight}_{p~c}$ indicates weight value for decorrelation pass $p$, channel $c$}}
\SetKwData{MONO}{mono output}
\SetKwData{FALSESTEREO}{false stereo}
\SetKwData{SUBBLOCKSIZE}{sub block size}
\SetKwData{ACTUALSIZEONELESS}{actual size 1 less}
\SetKwData{WEIGHTCOUNT}{weight count}
\SetKwData{TERMCOUNT}{term count}
\SetKwData{WEIGHTVAL}{weight value}
\SetKwData{WEIGHT}{weight}
\SetKw{AND}{and}
\tcc{read as many 8 bit weight values as possible}
\eIf{$\text{\ACTUALSIZEONELESS} = 0$}{
  \WEIGHTCOUNT $\leftarrow \text{\SUBBLOCKSIZE} \times 2$\;
}{
 \WEIGHTCOUNT $\leftarrow \text{\SUBBLOCKSIZE} \times 2 - 1$\;
}
\For{$i \leftarrow 0$ \emph{\KwTo}\WEIGHTCOUNT}{
  $\text{value}_i \leftarrow$ \READ 8 signed bits\;
  $\text{\WEIGHTVAL}_i \leftarrow\begin{cases}
\text{value}_i \times 2 ^ 3 + \left\lfloor\frac{\text{value}_i \times 2 ^ 3 + 2 ^ 6}{2 ^ 7}\right\rfloor & \text{if }\text{value}_i > 0 \\
0 & \text{if }\text{value}_i = 0 \\
\text{value}_i \times 2 ^ 3 & \text{if }\text{value}_i < 0
\end{cases}$\;
}
\BlankLine
\tcc{populate weight values by channel, in reverse order}
\eIf(\tcc*[f]{two channels}){$\text{\MONO} = 0$ \AND $\text{\FALSESTEREO} = 0$}{
  \ASSERT $\lfloor\WEIGHTCOUNT \div 2\rfloor \leq \TERMCOUNT$\;
  \For{$i \leftarrow 0$ \emph{\KwTo}$\lfloor\WEIGHTCOUNT \div 2\rfloor$}{
    $\text{\WEIGHT}_{(\TERMCOUNT - i - 1)~0} \leftarrow \text{\WEIGHTVAL}_{i \times 2}$\;
    $\text{\WEIGHT}_{(\TERMCOUNT - i - 1)~1} \leftarrow \text{\WEIGHTVAL}_{i \times 2 + 1}$\;
  }
  \For{$i \leftarrow \lfloor\WEIGHTCOUNT \div 2\rfloor$ \emph{\KwTo}\TERMCOUNT}{
    $\text{\WEIGHT}_{(\TERMCOUNT - i - 1)~0} \leftarrow 0$\;
    $\text{\WEIGHT}_{(\TERMCOUNT - i - 1)~1} \leftarrow 0$\;
  }
  \Return a \WEIGHT value per pass, per channel\;
}(\tcc*[f]{one channel}){
  \ASSERT $\WEIGHTCOUNT \leq \TERMCOUNT$\;
  \For{$i \leftarrow 0$ \emph{\KwTo}\WEIGHTCOUNT}{
    $\text{\WEIGHT}_{(\TERMCOUNT - i - 1)~0} \leftarrow \text{\WEIGHTVAL}_{i}$\;
  }
  \For{$i \leftarrow \WEIGHTCOUNT$ \emph{\KwTo}\TERMCOUNT}{
    $\text{\WEIGHT}_{(\TERMCOUNT - i - 1)~0} \leftarrow 0$\;
  }
  \Return a \WEIGHT value per pass\;
}
\EALGORITHM
}
\begin{figure}[h]
  \includegraphics{wavpack/figures/decorr_weights.pdf}
\end{figure}

\clearpage

\subsubsection{Reading Decorrelation Weights Example}
Given a 2 channel block containing 5 decorrelation terms:
\begin{figure}[h]
\includegraphics{wavpack/figures/decorr_weights_parse.pdf}
\end{figure}
\begin{center}
{\renewcommand{\arraystretch}{1.25}
\begin{tabular}{r|r|>{$}r<{$}}
$i$ & $\text{value}_i$ & \text{weight value}_i \\
\hline
0 & 6 & 6 \times 2 ^ 3 + \lfloor(6 \times 2 ^ 3 + 2 ^ 6) \div 2 ^ 7\rfloor = 48 \\
1 & 6 & 6 \times 2 ^ 3 + \lfloor(6 \times 2 ^ 3 + 2 ^ 6) \div 2 ^ 7\rfloor = 48 \\
2 & 6 & 6 \times 2 ^ 3 + \lfloor(6 \times 2 ^ 3 + 2 ^ 6) \div 2 ^ 7\rfloor = 48 \\
3 & 6 & 6 \times 2 ^ 3 + \lfloor(6 \times 2 ^ 3 + 2 ^ 6) \div 2 ^ 7\rfloor = 48 \\
4 & 4 & 4 \times 2 ^ 3 + \lfloor(4 \times 2 ^ 3 + 2 ^ 6) \div 2 ^ 7\rfloor = 32 \\
5 & 4 & 4 \times 2 ^ 3 + \lfloor(4 \times 2 ^ 3 + 2 ^ 6) \div 2 ^ 7\rfloor = 32 \\
6 & 6 & 6 \times 2 ^ 3 + \lfloor(6 \times 2 ^ 3 + 2 ^ 6) \div 2 ^ 7\rfloor = 48 \\
7 & 6 & 6 \times 2 ^ 3 + \lfloor(6 \times 2 ^ 3 + 2 ^ 6) \div 2 ^ 7\rfloor = 48 \\
8 & 2 & 2 \times 2 ^ 3 + \lfloor(2 \times 2 ^ 3 + 2 ^ 6) \div 2 ^ 7\rfloor = 16 \\
9 & 3 & 3 \times 2 ^ 3 + \lfloor(3 \times 2 ^ 3 + 2 ^ 6) \div 2 ^ 7\rfloor = 24 \\
\end{tabular}
\renewcommand{\arraystretch}{1.0}
}
\end{center}
\begin{center}
\begin{tabular}{>{$}r<{$}||>{$}r<{$}}
\text{weight}_{4~0} = \text{weight value}_0 = 48 &
\text{weight}_{4~1} = \text{weight value}_1 = 48 \\
\text{weight}_{3~0} = \text{weight value}_2 = 48 &
\text{weight}_{3~1} = \text{weight value}_3 = 48 \\
\text{weight}_{2~0} = \text{weight value}_4 = 32 &
\text{weight}_{2~1} = \text{weight value}_5 = 32 \\
\text{weight}_{1~0} = \text{weight value}_6 = 48 &
\text{weight}_{1~1} = \text{weight value}_7 = 48 \\
\text{weight}_{0~0} = \text{weight value}_8 = 16 &
\text{weight}_{0~1} = \text{weight value}_9 = 24 \\
\end{tabular}
\end{center}


\clearpage

%This work is licensed under the
%Creative Commons Attribution-Share Alike 3.0 United States License.
%To view a copy of this license, visit
%http://creativecommons.org/licenses/by-sa/3.0/us/ or send a letter to
%Creative Commons,
%171 Second Street, Suite 300,
%San Francisco, California, 94105, USA.

\subsection{Decoding Decorrelation Samples}
\label{wavpack:decode_decorrelation_samples}
{\relsize{-2}
\ALGORITHM{\VAR{mono output} and \VAR{false stereo} from block header, decorrelation terms, sub block size and data}{a list of signed decorrelation sample lists per channel per decorrelation term\footnote{\relsize{-1}$\text{sample}_{p~c~s}$ indicates the $s$th sample of decorrelation pass $p$ for channel $c$}}
\SetKwData{MONO}{mono output}
\SetKwData{FALSESTEREO}{false stereo}
\SetKwData{CHANNELS}{channel count}
\SetKwData{SAMPLE}{sample}
\SetKwData{TOTALSAMPLES}{sample count}
\SetKwData{TERMCOUNT}{term count}
\SetKwData{TERM}{term}
\SetKwFunction{EXP}{wv\_exp2}
\SetKw{KwDownTo}{downto}
\SetKw{AND}{and}
\eIf{$(\MONO = 0)$ \AND $(\FALSESTEREO = 0)$}{
  $\CHANNELS \leftarrow 2$\;
}{
  $\CHANNELS \leftarrow 1$\;
}
\For{$p \leftarrow \TERMCOUNT$ \emph{\KwDownTo}0}{
  \uIf(\tcc*[f]{2 samples per channel}){$17 \leq \text{\TERM}_p \leq 18$}{
    \eIf{$\text{sub block bytes remaining} \geq (\CHANNELS \times 4)$}{
      \For{$c \leftarrow 0$ \emph{\KwTo}\CHANNELS}{
        $\text{\SAMPLE}_{p~c~0} \leftarrow \text{read \EXP value}$\;
        $\text{\SAMPLE}_{p~c~1} \leftarrow \text{read \EXP value}$\;
      }
    }{
      \For{$c \leftarrow 0$ \emph{\KwTo}\CHANNELS}{
        $\text{\SAMPLE}_{p~c} \leftarrow \texttt{[0, 0]}$\;
      }
    }
  }
  \uElseIf(\tcc*[f]{"term" samples per channel}){$1 \leq \text{\TERM}_p \leq 8$}{
    \eIf{$\text{sub block bytes remaining} \geq (\CHANNELS \times \text{\TERM}_p \times 2)$}{
      \For{$s \leftarrow 0$ \emph{\KwTo}$\text{\TERM}_p$}{
        \For{$c \leftarrow 0$ \emph{\KwTo}\CHANNELS}{
          $\text{\SAMPLE}_{p~c~s} \leftarrow \text{read \EXP value}$\;
        }
      }
    }{
      \For{$s \leftarrow 0$ \emph{\KwTo}$\text{\TERM}_p$}{
        \For{$c \leftarrow 0$ \emph{\KwTo}\CHANNELS}{
          $\text{\SAMPLE}_{p~c~s} \leftarrow 0$\;
        }
      }
    }
  }
  \ElseIf(\tcc*[f]{1 sample per channel}){$-3 \leq \text{\TERM}_p \leq -1$}{
    \eIf{$\text{sub block bytes remaining} \geq 4$}{
      $\text{\SAMPLE}_{p~0~0} \leftarrow \text{read \EXP value}$\;
      $\text{\SAMPLE}_{p~1~0} \leftarrow \text{read \EXP value}$\;
    }{
      $\text{\SAMPLE}_{p~0~0} \leftarrow 0$\;
      $\text{\SAMPLE}_{p~1~0} \leftarrow 0$\;
    }
  }
}
\Return $\text{\SAMPLE}$ lists per pass, per channel\;
\EALGORITHM
}
\begin{figure}[h]
  \includegraphics{wavpack/figures/decorr_samples.pdf}
\end{figure}

\clearpage

\subsubsection{Reading wv\_exp2 Values}
\label{wavpack_wvexp2}
{\relsize{-1}
\ALGORITHM{2 bytes of sub block data}{a signed value}
\SetKwFunction{EXP}{wexp}
$value \leftarrow$ \READ 16 signed bits\;
\BlankLine
\uIf{$-32768 \leq value < -2304$}{
  \Return $-(\EXP(-value \bmod{256}) \times 2 ^ {\lfloor -value \div 2 ^ 8 \rfloor - 9})$\;
}
\uElseIf{$-2304 \leq value < 0$}{
  \Return $-\lfloor \EXP(-value \bmod{256}) \div 2 ^ {9 - \lfloor -value \div 2 ^ 8 \rfloor} \rfloor$\;
}
\uElseIf{$0 \leq value \leq 2304$}{
  \Return $\lfloor \EXP(value \bmod{256}) \div 2 ^ {9 - \lfloor value \div 2 ^ 8 \rfloor} \rfloor$\;
}
\ElseIf{$2304 < value \leq 32767$}{
  \Return $\EXP(value \bmod{256}) \times 2 ^ {\lfloor value \div 2 ^ 8 \rfloor - 9}$\;
}
\EALGORITHM
}
\par
\noindent
where \texttt{wexp}(\textit{x}) is defined from the following table:
\vskip .10in
\par
\noindent
{\relsize{-3}\ttfamily
\begin{tabular}{| c | c | c | c | c | c | c | c | c | c | c | c | c | c | c | c | c |}
\hline
& 0x?0 & 0x?1 & 0x?2 & 0x?3 & 0x?4 & 0x?5 & 0x?6 & 0x?7 & 0x?8 & 0x?9 & 0x?A & 0x?B & 0x?C & 0x?D & 0x?E & 0x?F \\
\hline
0x0? & 256 & 257 & 257 & 258 & 259 & 259 & 260 & 261 & 262 & 262 & 263 & 264 & 264 & 265 & 266 & 267 \\
0x1? & 267 & 268 & 269 & 270 & 270 & 271 & 272 & 272 & 273 & 274 & 275 & 275 & 276 & 277 & 278 & 278 \\
0x2? & 279 & 280 & 281 & 281 & 282 & 283 & 284 & 285 & 285 & 286 & 287 & 288 & 288 & 289 & 290 & 291 \\
0x3? & 292 & 292 & 293 & 294 & 295 & 296 & 296 & 297 & 298 & 299 & 300 & 300 & 301 & 302 & 303 & 304 \\
0x4? & 304 & 305 & 306 & 307 & 308 & 309 & 309 & 310 & 311 & 312 & 313 & 314 & 314 & 315 & 316 & 317 \\
0x5? & 318 & 319 & 320 & 321 & 321 & 322 & 323 & 324 & 325 & 326 & 327 & 328 & 328 & 329 & 330 & 331 \\
0x6? & 332 & 333 & 334 & 335 & 336 & 337 & 337 & 338 & 339 & 340 & 341 & 342 & 343 & 344 & 345 & 346 \\
0x7? & 347 & 348 & 349 & 350 & 350 & 351 & 352 & 353 & 354 & 355 & 356 & 357 & 358 & 359 & 360 & 361 \\
0x8? & 362 & 363 & 364 & 365 & 366 & 367 & 368 & 369 & 370 & 371 & 372 & 373 & 374 & 375 & 376 & 377 \\
0x9? & 378 & 379 & 380 & 381 & 382 & 383 & 384 & 385 & 386 & 387 & 388 & 389 & 391 & 392 & 393 & 394 \\
0xA? & 395 & 396 & 397 & 398 & 399 & 400 & 401 & 402 & 403 & 405 & 406 & 407 & 408 & 409 & 410 & 411 \\
0xB? & 412 & 413 & 415 & 416 & 417 & 418 & 419 & 420 & 421 & 422 & 424 & 425 & 426 & 427 & 428 & 429 \\
0xC? & 431 & 432 & 433 & 434 & 435 & 436 & 438 & 439 & 440 & 441 & 442 & 444 & 445 & 446 & 447 & 448 \\
0xD? & 450 & 451 & 452 & 453 & 454 & 456 & 457 & 458 & 459 & 461 & 462 & 463 & 464 & 466 & 467 & 468 \\
0xE? & 470 & 471 & 472 & 473 & 475 & 476 & 477 & 478 & 480 & 481 & 482 & 484 & 485 & 486 & 488 & 489 \\
0xF? & 490 & 492 & 493 & 494 & 496 & 497 & 498 & 500 & 501 & 502 & 504 & 505 & 506 & 508 & 509 & 511 \\
\hline
\end{tabular}
}

\subsubsection{Reading Decorrelation Samples Example}
Given a stereo block containing the sub-block:
\begin{figure}[h]
\includegraphics{wavpack/figures/decorr_samples_parse.pdf}
\end{figure}
\begin{center}
{\relsize{-2}
\begin{tabular}{r|r|r|>{$}r<{$}|>{$}r<{$}}
$p$ & $\text{term}_p$ & $s$ &
\text{sample}_{p~0~s} &
\text{sample}_{p~1~s} \\
\hline
4 & 18 & 0 &
-\lfloor \texttt{wexp}(1841 \bmod{256}) \div 2 ^ {9 - \lfloor 1841 \div 2 ^ 8 \rfloor} \rfloor = -73 &
\lfloor \EXP(1487 \bmod{256}) \div 2 ^ {9 - \lfloor 1487 \div 2 ^ 8 \rfloor} \rfloor = 28 \\
& & 1 &
-\lfloor \EXP(1865 \bmod{256}) \div 2 ^ {9 - \lfloor 1865 \div 2 ^ 8 \rfloor} \rfloor = -78 &
\lfloor \EXP(1459 \bmod{256}) \div 2 ^ {9 - \lfloor 1459 \div 2 ^ 8 \rfloor} \rfloor = 26 \\
\hline
3 & 18 & 0 & 0 & 0 \\
& & 1 & 0 & 0 \\
\hline
2 & 2 & 0 & 0 & 0 \\
& & 1 & 0 & 0 \\
\hline
1 & 17 & 0 & 0 & 0 \\
& & 1 & 0 & 0 \\
\hline
0 & 3 & 0 & 0 & 0 \\
& & 1 & 0 & 0 \\
& & 2 & 0 & 0 \\
\hline
\end{tabular}
\renewcommand{\arraystretch}{1.0}
}
\end{center}


\clearpage

%This work is licensed under the
%Creative Commons Attribution-Share Alike 3.0 United States License.
%To view a copy of this license, visit
%http://creativecommons.org/licenses/by-sa/3.0/us/ or send a letter to
%Creative Commons,
%171 Second Street, Suite 300,
%San Francisco, California, 94105, USA.

\subsection{Writing Entropy Variables}
\label{wavpack:write_entropy}
\ALGORITHM{3 entropy variables per channel, channel count}{entropy variables sub block data}
\SetKwData{ENTROPY}{entropy}
\SetKwFunction{WVLOG}{wv\_log2}
$\WVLOG(\text{\ENTROPY}_{0~0}) \rightarrow$ \WRITE 16 signed bits\;
$\WVLOG(\text{\ENTROPY}_{0~1}) \rightarrow$ \WRITE 16 signed bits\;
$\WVLOG(\text{\ENTROPY}_{0~2}) \rightarrow$ \WRITE 16 signed bits\;
\If{$\text{channel count} = 2$}{
  $\WVLOG(\text{\ENTROPY}_{1~0}) \rightarrow$ \WRITE 16 signed bits\;
  $\WVLOG(\text{\ENTROPY}_{1~1}) \rightarrow$ \WRITE 16 signed bits\;
  $\WVLOG(\text{\ENTROPY}_{1~2}) \rightarrow$ \WRITE 16 signed bits\;
}
\Return entropy variables sub block data\;
\EALGORITHM

\begin{figure}[h]
  \includegraphics{wavpack/figures/entropy_vars.pdf}
\end{figure}

\clearpage

\subsubsection{Writing Entropy Variables Example}

\begin{table}[h]
{\relsize{-2}
  \renewcommand{\arraystretch}{1.5}
\begin{tabular}{r|>{$}r<{$}|>{$}r<{$}|>{$}r<{$}}
  $\text{entropy}_{c~i}$ & $a$ & $c$ & \text{value}_{c~i} \\
  \hline
  118 &
  |118| + \lfloor |118| \div 2 ^ 9\rfloor = 118 &
  \lfloor\log_2(118)\rfloor + 1 = 7 &
  7 \times 2 ^ 8 + \texttt{wlog}((118 \times 2 ^ {9 - 7}) \bmod 256) = 2018 \\
  194 &
  |194| + \lfloor |194| \div 2 ^ 9\rfloor = 194 &
  \lfloor\log_2(194)\rfloor + 1 = 8 &
  8 \times 2 ^ 8 + \texttt{wlog}((194 \times 2 ^ {9 - 8}) \bmod 256) = 2202 \\
  322 &
  |322| + \lfloor |322| \div 2 ^ 9\rfloor = 322 &
  \lfloor\log_2(322)\rfloor + 1 = 9 &
  9 \times 2 ^ 8 + \LOG(\lfloor 322 \div 2 ^ {9 - 9}\rfloor \bmod 256) = 2389 \\
  \hline
  118 &
  |118| + \lfloor |118| \div 2 ^ 9\rfloor = 118 &
  \lfloor\log_2(118)\rfloor + 1 = 7 &
  7 \times 2 ^ 8 + \texttt{wlog}((118 \times 2 ^ {9 - 7}) \bmod 256) = 2018 \\
  176 &
  |176| + \lfloor |176| \div 2 ^ 9\rfloor = 176 &
  \lfloor\log_2(176)\rfloor + 1 = 8 &
  8 \times 2 ^ 8 + \texttt{wlog}((176 \times 2 ^ {9 - 8}) \bmod 256) = 2166 \\
  212 &
  |212| + \lfloor |212| \div 2 ^ 9\rfloor = 212 &
  \lfloor\log_2(212)\rfloor + 1 = 8 &
  8 \times 2 ^ 8 + \texttt{wlog}((212 \times 2 ^ {9 - 8}) \bmod 256) = 2234 \\
\end{tabular}
}
\end{table}
\begin{figure}[h]
  \includegraphics{wavpack/figures/entropy_vars_parse.pdf}
\end{figure}


\clearpage

%This work is licensed under the
%Creative Commons Attribution-Share Alike 3.0 United States License.
%To view a copy of this license, visit
%http://creativecommons.org/licenses/by-sa/3.0/us/ or send a letter to
%Creative Commons,
%171 Second Street, Suite 300,
%San Francisco, California, 94105, USA.

\subsection{Correlation Passes}
\label{wavpack:correlate_channels}
\ALGORITHM{a list of signed samples per channel; correlation terms, deltas, weights and samples}{a list of signed residuals per channel; correlation weights and samples for the next block}
\SetKwData{PASS}{pass}
\SetKwData{CHANNEL}{channel}
\SetKwData{TERMCOUNT}{term count}
\SetKwData{TERM}{term}
\SetKwData{DELTA}{delta}
\SetKwData{WEIGHT}{weight}
\SetKwData{SAMPLES}{sample}
\SetKw{KwDownTo}{downto}
\eIf{$\text{channel count} = 1$}{
  $\text{\PASS}_{-1~0} \leftarrow \text{\CHANNEL}_0$\;
  \For(\tcc*[f]{perform passes in reverse order}){$i \leftarrow 0$ \emph{\KwTo}\TERMCOUNT}{
    $p \leftarrow \TERMCOUNT - i - 1$\;
    $\left.\begin{tabular}{r}
      $\text{\PASS}_{i~0}$ \\
      $\text{\WEIGHT}_{p~0}$ \\
      $\text{\SAMPLES}_{p~0}$ \\
    \end{tabular}\right\rbrace \leftarrow$ \hyperref[wavpack:correlate_1ch]{correlate 1 channel $\text{\PASS}_{i - 1}$} using $\left\lbrace\begin{tabular}{l}
    $\text{\TERM}_{i}$ \\
    $\text{\DELTA}_{i}$ \\
    $\text{\WEIGHT}_{p~0}$ \\
    $\text{\SAMPLES}_{p~0}$ \\
    \end{tabular}\right.$\;
  }
  \Return $\text{\PASS}_{(\TERMCOUNT - 1)}$, updated $\text{\WEIGHT}$, updated $\text{\SAMPLES}$\;
}{
  $\text{\PASS}_{-1~0} \leftarrow \text{\CHANNEL}_0$\;
  $\text{\PASS}_{-1~1} \leftarrow \text{\CHANNEL}_1$\;
  \For(\tcc*[f]{perform passes in reverse order}){$i \leftarrow 0$ \emph{\KwTo}\TERMCOUNT}{
    $p \leftarrow \TERMCOUNT - i - 1$\;
    $\left.\begin{tabular}{r}
      $\text{\PASS}_{i~0}$ \\
      $\text{\PASS}_{i~1}$ \\
      $\text{\WEIGHT}_{p~0}$ \\
      $\text{\WEIGHT}_{p~1}$ \\
      $\text{\SAMPLES}_{p~0}$ \\
      $\text{\SAMPLES}_{p~1}$ \\
    \end{tabular}\right\rbrace \leftarrow$ \hyperref[wavpack:correlate_2ch]{correlate 2 channel $\text{\PASS}_{i - 1}$} using $\left\lbrace\begin{tabular}{l}
    $\text{\TERM}_{i}$ \\
    $\text{\DELTA}_{i}$ \\
    $\text{\WEIGHT}_{p~0}$ \\
    $\text{\WEIGHT}_{p~1}$ \\
    $\text{\SAMPLES}_{p~0}$ \\
    $\text{\SAMPLES}_{p~1}$ \\
    \end{tabular}\right.$\;
  }
  \Return $\text{\PASS}_{(\TERMCOUNT - 1)}$, updated $\text{\WEIGHT}$, updated $\text{\SAMPLES}$\;
}
\EALGORITHM

\clearpage

\subsection{1 Channel Correlation Pass}
\label{wavpack:correlate_1ch}
{\relsize{-1}
\ALGORITHM{a list of signed uncorrelated samples; correlation term, delta, weight and samples}{a list of signed correlated samples; updated weight and sample values}
\SetKwData{CORRELATED}{correlated}
\SetKwData{DECORRELATED}{uncorrelated}
\SetKwData{DECORRSAMPLE}{correlation sample}
\SetKwData{WEIGHT}{weight}
\SetKwData{DELTA}{delta}
\SetKwData{TEMP}{temp}
\SetKw{OR}{or}
\SetKw{XOR}{xor}
\SetKwFunction{APPLYWEIGHT}{apply\_weight}
\SetKwFunction{UPDATEWEIGHT}{update\_weight}
$\text{\WEIGHT}_0 \leftarrow$ decorrelation weight\;
\BlankLine
\uIf{$term = 18$}{
  $\text{\DECORRELATED}_{-2} \leftarrow \text{\DECORRSAMPLE}_1$\;
  $\text{\DECORRELATED}_{-1} \leftarrow \text{\DECORRSAMPLE}_0$\;
  \For{$i \leftarrow 0$ \emph{\KwTo}uncorrelated samples length}{
    $\text{\TEMP}_{i} \leftarrow \lfloor(3 \times \text{\DECORRELATED}_{i - 1} - \text{\DECORRELATED}_{i - 2}) \div 2 \rfloor$\;
    $\text{\CORRELATED}_i \leftarrow \text{\DECORRELATED}_i - \APPLYWEIGHT(\text{\WEIGHT}_i~,~\text{\TEMP}_{i})$\;
    $\text{\WEIGHT}_{i + 1} \leftarrow \text{\WEIGHT}_i + \UPDATEWEIGHT(\text{\TEMP}_{i}~,~\text{\CORRELATED}_i~,~\DELTA)$\;
  }
  \Return $\left\lbrace\begin{tabular}{l}
  \CORRELATED \\
  $\text{\WEIGHT}_i$ \\
  $\DECORRSAMPLE \leftarrow$ \texttt{[}$\text{\DECORRELATED}_{(i - 2)}$, $\text{\DECORRELATED}_{(i - 1)}$ \texttt{]} \\
  \end{tabular}\right.$\;
}
\uElseIf{$term = 17$}{
  $\text{\DECORRELATED}_{-2} \leftarrow \text{\DECORRSAMPLE}_1$\;
  $\text{\DECORRELATED}_{-1} \leftarrow \text{\DECORRSAMPLE}_0$\;
  \For{$i \leftarrow 0$ \emph{\KwTo}uncorrelated samples length}{
    $\text{\TEMP}_{i} \leftarrow 2 \times \text{\DECORRELATED}_{i - 1} - \text{\DECORRELATED}_{i - 2}$\;
    $\text{\CORRELATED}_i \leftarrow  \text{\DECORRELATED}_i - \APPLYWEIGHT(\text{\WEIGHT}_i~,~\text{\TEMP}_{i})$\;
    $\text{\WEIGHT}_{i + 1} \leftarrow \text{\WEIGHT}_i + \UPDATEWEIGHT(\text{\TEMP}_{i}~,~\text{\CORRELATED}_i~,~\DELTA)$\;
  }
  \Return $\left\lbrace\begin{tabular}{l}
  \CORRELATED \\
  $\text{\WEIGHT}_i$ \\
  $\DECORRSAMPLE \leftarrow$ \texttt{[}$\text{\DECORRELATED}_{(i - 2)}$, $\text{\DECORRELATED}_{(i - 1)}$ \texttt{]} \\
  \end{tabular}\right.$\;
}
\uElseIf{$1 \leq term \leq 8$}{
  \For{$i \leftarrow 0$ \emph{\KwTo}term}{
    $\text{\DECORRELATED}_{i - \text{term}} \leftarrow \text{\DECORRSAMPLE}_i$\;
  }
  \For{$i \leftarrow 0$ \emph{\KwTo}correlated samples length}{
    $\text{\CORRELATED}_i \leftarrow  \text{\DECORRELATED}_i - \APPLYWEIGHT(\text{\WEIGHT}_i~,~\text{\DECORRELATED}_{i - \text{term}})$\;
    $\text{\WEIGHT}_{i + 1} \leftarrow \text{\WEIGHT}_i + \UPDATEWEIGHT(\text{\DECORRELATED}_{i - \text{term}}~,~\text{\CORRELATED}_i~,~\DELTA)$\;
  }
  \Return $\left\lbrace\begin{tabular}{l}
  \CORRELATED \\
  $\text{\WEIGHT}_i$ \\
  $\DECORRSAMPLE \leftarrow$ last $term$ \DECORRELATED samples \\
  \end{tabular}\right.$\;
}
\Else{
  invalid decorrelation term\;
}
\EALGORITHM
\par
\noindent
\begin{align*}
\intertext{where \texttt{apply\_weight} is defined as:}
\texttt{apply\_weight}(weight~,~sample) &= \left\lfloor\frac{weight \times sample + 2 ^ 9}{2 ^ {10}}\right\rfloor \\
\intertext{and \texttt{update\_weight} is defined as:}
\texttt{update\_weight}(source~,~result~,~delta) &=
\begin{cases}
0 & \text{ if } source = 0 \text{ or } result = 0 \\
delta & \text{ if } (source \textbf{ xor } result ) \geq 0 \\
-delta & \text{ if } (source \textbf{ xor } result) < 0
\end{cases}
\end{align*}
}

\clearpage

\subsection{2 Channel Correlation Pass}
\label{wavpack:correlate_2ch}
{\relsize{-1}
\ALGORITHM{2 lists of signed uncorrelated samples; correlation term and delta, 2 correlation weights, 2 lists of correlation samples}{2 lists of signed correlated samples; updated weight and sample values per channel}
\SetKwData{TERM}{term}
\SetKwData{DELTA}{delta}
\SetKwData{CORRELATED}{correlated}
\SetKwData{DECORRELATED}{uncorrelated}
\SetKwData{DECORRSAMPLE}{correlation sample}
\SetKwData{WEIGHT}{weight}
\SetKw{OR}{or}
\SetKw{XOR}{xor}
\SetKwFunction{MIN}{min}
\SetKwFunction{MAX}{max}
\SetKwFunction{APPLYWEIGHT}{apply\_weight}
\SetKwFunction{UPDATEWEIGHT}{update\_weight}
\uIf{$(17 \leq \TERM \leq 18)$ \OR $(1 \leq \TERM \leq 8)$}{
  $\left.\begin{tabular}{r}
    $\text{\CORRELATED}_{0}$ \\
    $\text{\WEIGHT}_{0}$ \\
    $\text{\DECORRSAMPLE}_{0}$ \\
  \end{tabular}\right\rbrace \leftarrow$ \hyperref[wavpack:correlate_1ch]{correlate 1 channel $\text{\DECORRELATED}_{0}$} using $\left\lbrace\begin{tabular}{l}
  $\text{\TERM}$ \\
  $\text{\DELTA}$ \\
  $\text{\WEIGHT}_{0}$ \\
  $\text{\DECORRSAMPLE}_{0}$ \\
  \end{tabular}\right.$\;
  $\left.\begin{tabular}{r}
    $\text{\CORRELATED}_{1}$ \\
    $\text{\WEIGHT}_{1}$ \\
    $\text{\DECORRSAMPLE}_{1}$ \\
  \end{tabular}\right\rbrace \leftarrow$ \hyperref[wavpack:correlate_1ch]{correlate 1 channel $\text{\DECORRELATED}_{1}$} using $\left\lbrace\begin{tabular}{l}
  $\text{\TERM}$ \\
  $\text{\DELTA}$ \\
  $\text{\WEIGHT}_{1}$ \\
  $\text{\DECORRSAMPLE}_{1}$ \\
  \end{tabular}\right.$\;
  \Return $\left\lbrace\begin{tabular}{l}
  $\text{\CORRELATED}$ \\
  $\text{\WEIGHT}$ \\
  $\text{\DECORRSAMPLE}$ \\
  \end{tabular}\right.$\;
}
\uElseIf{$-3 \leq \TERM \leq -1$}{
  $\text{\WEIGHT}_{0~0} \leftarrow$ correlation weight 0\;
  $\text{\WEIGHT}_{1~0} \leftarrow$ correlation weight 1\;
  $\text{\DECORRELATED}_{0~-1} \leftarrow \text{\DECORRSAMPLE}_{1~0}$\;
  $\text{\DECORRELATED}_{1~-1} \leftarrow \text{\DECORRSAMPLE}_{0~0}$\;
  \uIf{$\TERM = -1$}{
    \For{$i \leftarrow 0$ \emph{\KwTo}uncorrelated samples length}{
      $\text{\CORRELATED}_{0~i} \leftarrow  \text{\DECORRELATED}_{0~i} - \APPLYWEIGHT(\text{\WEIGHT}_{0~i}~,~\text{\DECORRELATED}_{1~(i - 1)})$\;
      $\text{\CORRELATED}_{1~i} \leftarrow \text{\DECORRELATED}_{1~i} - \APPLYWEIGHT(\text{\WEIGHT}_{1~i}~,~\text{\DECORRELATED}_{0~i})$\;
      $\text{\WEIGHT}_{0~(i + 1)} \leftarrow \text{\WEIGHT}_{0~i} + \UPDATEWEIGHT(\text{\DECORRELATED}_{1~(i - 1)}~,~\text{\CORRELATED}_{0~i}~,~\DELTA)$\;
      $\text{\WEIGHT}_{1~(i + 1)} \leftarrow \text{\WEIGHT}_{1~i} + \UPDATEWEIGHT(\text{\DECORRELATED}_{0~i}~,~\text{\CORRELATED}_{1~i}~,~\DELTA)$\;
      $\text{\WEIGHT}_{0~(i + 1)} \leftarrow \MAX(\MIN(\text{\WEIGHT}_{0~(i + 1)}~,~1024)~,~-1024)$\;
      $\text{\WEIGHT}_{1~(i + 1)} \leftarrow \MAX(\MIN(\text{\WEIGHT}_{1~(i + 1)}~,~1024)~,~-1024)$\;
    }
  }
  \uElseIf{$\TERM = -2$}{
    \For{$i \leftarrow 0$ \emph{\KwTo}uncorrelated samples length}{
      $\text{\CORRELATED}_{0~i} \leftarrow \text{\DECORRELATED}_{0~i} - \APPLYWEIGHT(\text{\WEIGHT}_{0~i}~,~\text{\DECORRELATED}_{1~i})$\;
      $\text{\CORRELATED}_{1~i} \leftarrow \text{\DECORRELATED}_{1~i} - \APPLYWEIGHT(\text{\WEIGHT}_{1~i}~,~\text{\DECORRELATED}_{0~({i - 1})})$\;
      $\text{\WEIGHT}_{0~(i + 1)} \leftarrow \text{\WEIGHT}_{0~i} + \UPDATEWEIGHT(\text{\DECORRELATED}_{1~i}~,~\text{\CORRELATED}_{0~i}~,~\DELTA)$\;
      $\text{\WEIGHT}_{1~(i + 1)} \leftarrow \text{\WEIGHT}_{1~i} + \UPDATEWEIGHT(\text{\DECORRELATED}_{0~(i - 1)}~,~\text{\CORRELATED}_{1~i}~,~\DELTA)$\;
      $\text{\WEIGHT}_{0~(i + 1)} \leftarrow \MAX(\MIN(\text{\WEIGHT}_{0~(i + 1)}~,~1024)~,~-1024)$\;
      $\text{\WEIGHT}_{1~(i + 1)} \leftarrow \MAX(\MIN(\text{\WEIGHT}_{1~(i + 1)}~,~1024)~,~-1024)$\;
    }
  }
  \ElseIf{$\TERM = -3$}{
    \For{$i \leftarrow 0$ \emph{\KwTo}uncorrelated samples length}{
      $\text{\CORRELATED}_{0~i} \leftarrow \text{\DECORRELATED}_{0~i} - \APPLYWEIGHT(\text{\WEIGHT}_{0~i}~,~\text{\DECORRELATED}_{1~(i - 1)})$\;
      $\text{\CORRELATED}_{1~i} \leftarrow \text{\DECORRELATED}_{1~i} - \APPLYWEIGHT(\text{\WEIGHT}_{1~i}~,~\text{\DECORRELATED}_{0~(i - 1)})$\;
      $\text{\WEIGHT}_{0~(i + 1)} \leftarrow \text{\WEIGHT}_{0~i} + \UPDATEWEIGHT(\text{\DECORRELATED}_{1~(i - 1)}~,~\text{\CORRELATED}_{0~i}~,~\DELTA)$\;
      $\text{\WEIGHT}_{1~(i + 1)} \leftarrow \text{\WEIGHT}_{1~i} + \UPDATEWEIGHT(\text{\DECORRELATED}_{0~(i - 1)}~,~\text{\CORRELATED}_{1~i}~,~\DELTA)$\;
      $\text{\WEIGHT}_{0~(i + 1)} \leftarrow \MAX(\MIN(\text{\WEIGHT}_{0~(i + 1)}~,~1024)~,~-1024)$\;
      $\text{\WEIGHT}_{1~(i + 1)} \leftarrow \MAX(\MIN(\text{\WEIGHT}_{1~(i + 1)}~,~1024)~,~-1024)$\;
    }
  }
  \Return $\left\lbrace\begin{tabular}{l}
  $\text{\CORRELATED}$ \\
  $\text{\WEIGHT}_{0~i}$ \\
  $\text{\WEIGHT}_{1~i}$ \\
  $\text{\DECORRSAMPLE}_{0~0} \leftarrow \text{\DECORRELATED}_{1~i}$ \\
  $\text{\DECORRSAMPLE}_{1~0} \leftarrow \text{\DECORRELATED}_{0~i}$ \\
  \end{tabular}\right.$\;
}
\Else{
  invalid decorrelation term\;
}
\EALGORITHM
}

\clearpage

\subsection{Channel Correlation Example}
\begin{figure}[h]
{\relsize{-1}
  \subfloat{
    \begin{tabular}{|r|r|r|}
      \multicolumn{3}{c}{Correlation Terms} \\
      \hline
      $p$ & $\textsf{term}_p$ & $\textsf{delta}_p$ \\
      \hline
      0 & 3 & 2 \\
      1 & 17 & 2 \\
      2 & 2 & 2 \\
      3 & 18 & 2 \\
      4 & 18 & 2 \\
      \hline
    \end{tabular}
  }
  \subfloat{
    \begin{tabular}{|r|r|r|}
      \multicolumn{3}{c}{Correlation Weights} \\
      \hline
      $p$ & $\textsf{weight}_{p~0}$ & $\textsf{weight}_{p~1}$ \\
      \hline
      0 & 16 & 24 \\
      1 & 48 & 48 \\
      2 & 32 & 32 \\
      3 & 48 & 48 \\
      4 & 48 & 48 \\
      \hline
    \end{tabular}
  }
  \subfloat{
    \begin{tabular}{|r|r|r|}
      \multicolumn{3}{c}{Correlation Samples} \\
      \hline
      $p$ & $\textsf{sample}_{p~0~s}$ & $\textsf{sample}_{p~1~s}$ \\
      \hline
      0 & \texttt{[0, 0, 0]} & \texttt{[0, 0, 0]} \\
      1 & \texttt{[0, 0]} & \texttt{[0, 0]} \\
      2 & \texttt{[0, 0]} & \texttt{[0, 0]} \\
      3 & \texttt{[0, 0]} & \texttt{[0, 0]} \\
      4 & \texttt{[-73, -78]} & \texttt{[28, 26]} \\
      \hline
    \end{tabular}
  }
}
\end{figure}
\par
\noindent
we combine them into a single set of arguments for each correlation pass:
\begin{table}[h]
{\relsize{-1}
  \begin{tabular}{|r|r|r|r|r|r|}
    \hline
    & $\textbf{pass}_0$ & $\textbf{pass}_1$ & $\textbf{pass}_2$ &
    $\textbf{pass}_3$ & $\textbf{pass}_3$ \\
    \hline
    $\textsf{term}_p$ & 18 & 18 & 2 & 17 & 3 \\
    $\textsf{delta}_p$ & 2 & 2 & 2 & 2 & 2 \\
    $\textsf{weight}_{p~0}$ & 48 & 48 & 32 & 48 & 16 \\
    $\textsf{sample}_{p~0~s}$ & \texttt{[-73, -78]} & \texttt{[0, 0]} &
    \texttt{[0, 0]} & \texttt{[0, 0]} & \texttt{[0, 0, 0]} \\
    $\textsf{weight}_{p~1}$ & 48 & 48 & 32 & 48 & 24 \\
    $\textsf{sample}_{p~1~s}$ & \texttt{[28, 26]} & \texttt{[0, 0]} &
    \texttt{[0, 0]} & \texttt{[0, 0]} & \texttt{[0, 0, 0]} \\
    \hline
  \end{tabular}
}
\end{table}
\par
\noindent
which we apply to the residuals from the bitstream sub-block:
\par
\noindent
{\relsize{-1}
  \begin{tabular}{|r|r|r|r|r|r|}
    \hline
    $\textsf{channel}_{0~i}$ &
    after $\textbf{pass}_0$ &
    after $\textbf{pass}_1$ &
    after $\textbf{pass}_2$ &
    after $\textbf{pass}_3$ &
    after $\textbf{pass}_4$ \\
    \hline
    -64 & -61 & -61 & -61 & -61 & -61 \\
    -46 & -43 & -39 & -39 & -33 & -33 \\
    -25 & -23 & -21 & -19 & -18 & -18 \\
    -3 & -2 & -1 & 0 & 0 & 1 \\
    20 & 20 & 20 & 21 & 20 & 20 \\
    41 & 39 & 37 & 37 & 35 & 35 \\
    60 & 57 & 54 & 53 & 50 & 50 \\
    75 & 71 & 67 & 66 & 62 & 62 \\
    85 & 80 & 75 & 73 & 68 & 68 \\
    90 & 84 & 79 & 77 & 72 & 71 \\
    \hline
    \hline
    $\textsf{channel}_{1~i}$ &
    after $\textbf{pass}_0$ &
    after $\textbf{pass}_1$ &
    after $\textbf{pass}_2$ &
    after $\textbf{pass}_3$ &
    after $\textbf{pass}_4$ \\
    \hline
    32 & 31 & 31 & 31 & 31 & 31 \\
    39 & 37 & 35 & 35 & 32 & 32 \\
    43 & 41 & 39 & 38 & 36 & 36 \\
    45 & 43 & 41 & 40 & 38 & 37 \\
    44 & 41 & 39 & 38 & 36 & 35 \\
    40 & 38 & 36 & 34 & 32 & 31 \\
    34 & 32 & 30 & 28 & 26 & 25 \\
    25 & 23 & 21 & 20 & 19 & 18 \\
    15 & 14 & 13 & 12 & 11 & 10 \\
    4 & 3 & 2 & 1 & 1 & 0 \\
    \hline
  \end{tabular}
}
\par
\noindent
Resulting in final correlated samples:
\newline
\begin{tabular}{rr}
$\textsf{residual}_0$ : & \texttt{[-61,~-33,~-18,~~1,~20,~35,~50,~62,~68,~71]} \\
$\textsf{residual}_1$ : & \texttt{[~31,~~32,~~36,~37,~35,~31,~25,~18,~10,~~0]} \\
\end{tabular}

\clearpage

{\relsize{-2}
\begin{tabular}{r||r|>{$}r<{$}|>{$}r<{$}|>{$}r<{$}|>{$}r<{$}}
& $i$ & \textsf{uncorrelated}_i & \textsf{temp}_i & \textsf{correlated}_i & \textsf{weight}_{i + 1} \\
\hline
%%START
\multirow{10}{1em}{\begin{sideways}$\textbf{pass}_0$ - term 18\end{sideways}}
& 0 & -64 &
\lfloor(3 \times -73 + 78) \div 2\rfloor = -71 &
-64 - \lfloor(48 \times -71 + 2 ^ 9) \div 2 ^ {10}\rfloor = -61 &
48 + 2 = 50
\\
& 1 & -46 &
\lfloor(3 \times -64 + 73) \div 2\rfloor = -60 &
-46 - \lfloor(50 \times -60 + 2 ^ 9) \div 2 ^ {10}\rfloor = -43 &
50 + 2 = 52
\\
& 2 & -25 &
\lfloor(3 \times -46 + 64) \div 2\rfloor = -37 &
-25 - \lfloor(52 \times -37 + 2 ^ 9) \div 2 ^ {10}\rfloor = -23 &
52 + 2 = 54
\\
& 3 & -3 &
\lfloor(3 \times -25 + 46) \div 2\rfloor = -15 &
-3 - \lfloor(54 \times -15 + 2 ^ 9) \div 2 ^ {10}\rfloor = -2 &
54 + 2 = 56
\\
& 4 & 20 &
\lfloor(3 \times -3 + 25) \div 2\rfloor = 8 &
20 - \lfloor(56 \times 8 + 2 ^ 9) \div 2 ^ {10}\rfloor = 20 &
56 + 2 = 58
\\
& 5 & 41 &
\lfloor(3 \times 20 + 3) \div 2\rfloor = 31 &
41 - \lfloor(58 \times 31 + 2 ^ 9) \div 2 ^ {10}\rfloor = 39 &
58 + 2 = 60
\\
& 6 & 60 &
\lfloor(3 \times 41 - 20) \div 2\rfloor = 51 &
60 - \lfloor(60 \times 51 + 2 ^ 9) \div 2 ^ {10}\rfloor = 57 &
60 + 2 = 62
\\
& 7 & 75 &
\lfloor(3 \times 60 - 41) \div 2\rfloor = 69 &
75 - \lfloor(62 \times 69 + 2 ^ 9) \div 2 ^ {10}\rfloor = 71 &
62 + 2 = 64
\\
& 8 & 85 &
\lfloor(3 \times 75 - 60) \div 2\rfloor = 82 &
85 - \lfloor(64 \times 82 + 2 ^ 9) \div 2 ^ {10}\rfloor = 80 &
64 + 2 = 66
\\
& 9 & 90 &
\lfloor(3 \times 85 - 75) \div 2\rfloor = 90 &
90 - \lfloor(66 \times 90 + 2 ^ 9) \div 2 ^ {10}\rfloor = 84 &
66 + 2 = 68
\\
\hline
\hline
\multirow{10}{1em}{\begin{sideways}$\textbf{pass}_1$ - term 18\end{sideways}}
& 0 & -61 &
\lfloor(3 \times 0 - 0) \div 2\rfloor = 0 &
-61 - \lfloor(48 \times 0 + 2 ^ 9) \div 2 ^ {10}\rfloor = -61 &
48 + 0 = 48
\\
& 1 & -43 &
\lfloor(3 \times -61 - 0) \div 2\rfloor = -92 &
-43 - \lfloor(48 \times -92 + 2 ^ 9) \div 2 ^ {10}\rfloor = -39 &
48 + 2 = 50
\\
& 2 & -23 &
\lfloor(3 \times -43 + 61) \div 2\rfloor = -34 &
-23 - \lfloor(50 \times -34 + 2 ^ 9) \div 2 ^ {10}\rfloor = -21 &
50 + 2 = 52
\\
& 3 & -2 &
\lfloor(3 \times -23 + 43) \div 2\rfloor = -13 &
-2 - \lfloor(52 \times -13 + 2 ^ 9) \div 2 ^ {10}\rfloor = -1 &
52 + 2 = 54
\\
& 4 & 20 &
\lfloor(3 \times -2 + 23) \div 2\rfloor = 8 &
20 - \lfloor(54 \times 8 + 2 ^ 9) \div 2 ^ {10}\rfloor = 20 &
54 + 2 = 56
\\
& 5 & 39 &
\lfloor(3 \times 20 + 2) \div 2\rfloor = 31 &
39 - \lfloor(56 \times 31 + 2 ^ 9) \div 2 ^ {10}\rfloor = 37 &
56 + 2 = 58
\\
& 6 & 57 &
\lfloor(3 \times 39 - 20) \div 2\rfloor = 48 &
57 - \lfloor(58 \times 48 + 2 ^ 9) \div 2 ^ {10}\rfloor = 54 &
58 + 2 = 60
\\
& 7 & 71 &
\lfloor(3 \times 57 - 39) \div 2\rfloor = 66 &
71 - \lfloor(60 \times 66 + 2 ^ 9) \div 2 ^ {10}\rfloor = 67 &
60 + 2 = 62
\\
& 8 & 80 &
\lfloor(3 \times 71 - 57) \div 2\rfloor = 78 &
80 - \lfloor(62 \times 78 + 2 ^ 9) \div 2 ^ {10}\rfloor = 75 &
62 + 2 = 64
\\
& 9 & 84 &
\lfloor(3 \times 80 - 71) \div 2\rfloor = 84 &
84 - \lfloor(64 \times 84 + 2 ^ 9) \div 2 ^ {10}\rfloor = 79 &
64 + 2 = 66
\\
\hline
\hline
\multirow{10}{1em}{\begin{sideways}$\textbf{pass}_2$ - term 2\end{sideways}}
& 0 & -61 & &
-61 - \lfloor(32 \times 0 + 2 ^ 9) \div 2 ^ {10}\rfloor = -61 &
32 + 0 = 32
\\
& 1 & -39 & &
-39 - \lfloor(32 \times 0 + 2 ^ 9) \div 2 ^ {10}\rfloor = -39 &
32 + 0 = 32
\\
& 2 & -21 & &
-21 - \lfloor(32 \times -61 + 2 ^ 9) \div 2 ^ {10}\rfloor = -19 &
32 + 2 = 34
\\
& 3 & -1 & &
-1 - \lfloor(34 \times -39 + 2 ^ 9) \div 2 ^ {10}\rfloor = 0 &
34 + 0 = 34
\\
& 4 & 20 & &
20 - \lfloor(34 \times -21 + 2 ^ 9) \div 2 ^ {10}\rfloor = 21 &
34 - 2 = 32
\\
& 5 & 37 & &
37 - \lfloor(32 \times -1 + 2 ^ 9) \div 2 ^ {10}\rfloor = 37 &
32 - 2 = 30
\\
& 6 & 54 & &
54 - \lfloor(30 \times 20 + 2 ^ 9) \div 2 ^ {10}\rfloor = 53 &
30 + 2 = 32
\\
& 7 & 67 & &
67 - \lfloor(32 \times 37 + 2 ^ 9) \div 2 ^ {10}\rfloor = 66 &
32 + 2 = 34
\\
& 8 & 75 & &
75 - \lfloor(34 \times 54 + 2 ^ 9) \div 2 ^ {10}\rfloor = 73 &
34 + 2 = 36
\\
& 9 & 79 & &
79 - \lfloor(36 \times 67 + 2 ^ 9) \div 2 ^ {10}\rfloor = 77 &
36 + 2 = 38
\\
\hline
\hline
\multirow{10}{1em}{\begin{sideways}$\textbf{pass}_3$ - term 17\end{sideways}}
& 0 & -61 &
2 \times 0 - 0 = 0 &
-61 - \lfloor(48 \times 0 + 2 ^ 9) \div 2 ^ {10}\rfloor = -61 &
48 + 0 = 48
\\
& 1 & -39 &
2 \times -61 - 0 = -122 &
-39 - \lfloor(48 \times -122 + 2 ^ 9) \div 2 ^ {10}\rfloor = -33 &
48 + 2 = 50
\\
& 2 & -19 &
2 \times -39 + 61 = -17 &
-19 - \lfloor(50 \times -17 + 2 ^ 9) \div 2 ^ {10}\rfloor = -18 &
50 + 2 = 52
\\
& 3 & 0 &
2 \times -19 + 39 = 1 &
0 - \lfloor(52 \times 1 + 2 ^ 9) \div 2 ^ {10}\rfloor = 0 &
52 + 0 = 52
\\
& 4 & 21 &
2 \times 0 + 19 = 19 &
21 - \lfloor(52 \times 19 + 2 ^ 9) \div 2 ^ {10}\rfloor = 20 &
52 + 2 = 54
\\
& 5 & 37 &
2 \times 21 - 0 = 42 &
37 - \lfloor(54 \times 42 + 2 ^ 9) \div 2 ^ {10}\rfloor = 35 &
54 + 2 = 56
\\
& 6 & 53 &
2 \times 37 - 21 = 53 &
53 - \lfloor(56 \times 53 + 2 ^ 9) \div 2 ^ {10}\rfloor = 50 &
56 + 2 = 58
\\
& 7 & 66 &
2 \times 53 - 37 = 69 &
66 - \lfloor(58 \times 69 + 2 ^ 9) \div 2 ^ {10}\rfloor = 62 &
58 + 2 = 60
\\
& 8 & 73 &
2 \times 66 - 53 = 79 &
73 - \lfloor(60 \times 79 + 2 ^ 9) \div 2 ^ {10}\rfloor = 68 &
60 + 2 = 62
\\
& 9 & 77 &
2 \times 73 - 66 = 80 &
77 - \lfloor(62 \times 80 + 2 ^ 9) \div 2 ^ {10}\rfloor = 72 &
62 + 2 = 64
\\
\hline
\hline
\multirow{10}{1em}{\begin{sideways}$\textbf{pass}_4$ - term 3\end{sideways}}
& 0 & -61 & &
-61 - \lfloor(16 \times 0 + 2 ^ 9) \div 2 ^ {10}\rfloor = -61 &
16 + 0 = 16
\\
& 1 & -33 & &
-33 - \lfloor(16 \times 0 + 2 ^ 9) \div 2 ^ {10}\rfloor = -33 &
16 + 0 = 16
\\
& 2 & -18 & &
-18 - \lfloor(16 \times 0 + 2 ^ 9) \div 2 ^ {10}\rfloor = -18 &
16 + 0 = 16
\\
& 3 & 0 & &
0 - \lfloor(16 \times -61 + 2 ^ 9) \div 2 ^ {10}\rfloor = 1 &
16 - 2 = 14
\\
& 4 & 20 & &
20 - \lfloor(14 \times -33 + 2 ^ 9) \div 2 ^ {10}\rfloor = 20 &
14 - 2 = 12
\\
& 5 & 35 & &
35 - \lfloor(12 \times -18 + 2 ^ 9) \div 2 ^ {10}\rfloor = 35 &
12 - 2 = 10
\\
& 6 & 50 & &
50 - \lfloor(10 \times 0 + 2 ^ 9) \div 2 ^ {10}\rfloor = 50 &
10 + 0 = 10
\\
& 7 & 62 & &
62 - \lfloor(10 \times 20 + 2 ^ 9) \div 2 ^ {10}\rfloor = 62 &
10 + 2 = 12
\\
& 8 & 68 & &
68 - \lfloor(12 \times 35 + 2 ^ 9) \div 2 ^ {10}\rfloor = 68 &
12 + 2 = 14
\\
& 9 & 72 & &
72 - \lfloor(14 \times 50 + 2 ^ 9) \div 2 ^ {10}\rfloor = 71 &
14 + 2 = 16
\\
%%END
\end{tabular}
}
\begin{center}
$\text{channel}_0$ correlation passes
\end{center}


\clearpage

%This work is licensed under the
%Creative Commons Attribution-Share Alike 3.0 United States License.
%To view a copy of this license, visit
%http://creativecommons.org/licenses/by-sa/3.0/us/ or send a letter to
%Creative Commons,
%171 Second Street, Suite 300,
%San Francisco, California, 94105, USA.

\subsection{Encoding Bitstream}
\label{wavpack:write_bitstream}
{\relsize{-1}
\ALGORITHM{channel count, entropy values, a list of signed residual values per channel}{bitstream sub block data}
\SetKwData{UNDEFINED}{undefined}
\SetKwData{RESIDUAL}{residual}
\SetKwData{ENTROPY}{entropy}
\SetKwData{BLOCKSAMPLES}{block samples}
\SetKwData{CHANNELCOUNT}{channel count}
\SetKwData{ZEROES}{zeroes}
\SetKwData{OFFSET}{offset}
\SetKwData{ADD}{add}
\SetKwData{SIGN}{sign}
\SetKw{AND}{and}
\SetKw{OR}{or}
\SetKw{IS}{is}
\SetKwFunction{FLUSH}{flush}
\SetKwFunction{UNARYUNDEFINED}{unary\_undefined}
\SetKwFunction{ENCODERESIDUAL}{encode\_residual}
\SetKwFunction{UNARY}{unary}
\SetKwFunction{WRITERESIDUAL}{write\_residual}
$u_{(-2)} \leftarrow \text{\UNDEFINED}$\;
$\text{\ZEROES}_{(-1)} \leftarrow m_{(-1)} \leftarrow \text{\OFFSET}_{(-1)} \leftarrow \text{\ADD}_{(-1)} \leftarrow \text{\SIGN}_{(-1)} \leftarrow \text{\UNDEFINED}$\tcc*[r]{buffered $\text{residual}_{i - 1}$}
$i \leftarrow 0$\;
\BlankLine
\While{$i < (\text{\BLOCKSAMPLES} \times \text{\CHANNELCOUNT})$}{
  $r_i \leftarrow \text{\RESIDUAL}_{(i \bmod \text{\CHANNELCOUNT})~\lfloor i \div \text{\CHANNELCOUNT}\rfloor}$\;
  \eIf{$(\text{\ENTROPY}_{0~0} < 2)$ \AND $(\text{\ENTROPY}_{1~0} < 2)$ \AND $\UNARYUNDEFINED(u_{i - 2}~,~m_{i - 1})$}{
    \eIf(\tcc*[f]{in a block of zeroes}){$(\text{\ZEROES}_{i - 1} \neq \text{\UNDEFINED})$ \AND $(m_{i - 1} = \text{\UNDEFINED})$}{
      \eIf(\tcc*[f]{continue block of zeroes}){$r_i = 0$}{
        $\text{\ZEROES}_i \leftarrow \text{\ZEROES}_{i - 1} + 1$\;
        $(u_{i - 1}~,~m_i~,~\text{\OFFSET}_i~,~\text{\ADD}_i~,~\text{\SIGN}_i) \leftarrow (u_{i - 2}~,~m_{i - 1}~,~\text{\OFFSET}_{i - 1}~,~\text{\ADD}_{i - 1}~,~\text{\SIGN}_{i - 1})$\;
      }(\tcc*[f]{end block of zeroes}){
        $\text{\ZEROES}_i \leftarrow \text{\ZEROES}_{i - 1}$\;
        $(m_i~,~\text{\OFFSET}_i~,~\text{\ADD}_i~,~\text{\SIGN}_i) \leftarrow \hyperref[wavpack:encode_residual]{\ENCODERESIDUAL(r_i~,~\text{\ENTROPY}_{(i \bmod \text{\CHANNELCOUNT})})}$\;
      }
    }(\tcc*[f]{start a new block of zeroes}){
      \eIf{$r_i = 0$}{
        $\text{\ZEROES}_i \leftarrow 1$\;
        $m_i \leftarrow \text{\OFFSET}_i \leftarrow \text{\ADD}_i \leftarrow \text{\SIGN}_i \leftarrow \text{\UNDEFINED}$\;
        $u_{i - 1} \leftarrow \hyperref[wavpack:flush_residual]{\FLUSH(u_{i - 2}~,~\text{\ZEROES}_{i - 1}~,~m_{i - 1}~,~\text{\OFFSET}_{i - 1}~,~\text{\ADD}_{i - 1}~,~\text{\SIGN}_{i - 1}~,~0)}$\;
        $\text{\ENTROPY}_{0~0} \leftarrow \text{\ENTROPY}_{0~1} \leftarrow \text{\ENTROPY}_{0~2} \leftarrow \text{\ENTROPY}_{1~0} \leftarrow \text{\ENTROPY}_{1~1} \leftarrow \text{\ENTROPY}_{1~2} \leftarrow 0$\;
      }(\tcc*[f]{false-alarm block of zeroes}){
        $\text{\ZEROES}_i \leftarrow 0$\;
        $(m_i~,~\text{\OFFSET}_i~,~\text{\ADD}_i~,~\text{\SIGN}_i) \leftarrow \hyperref[wavpack:encode_residual]{\ENCODERESIDUAL(r_i~,~\text{\ENTROPY}_{(i \bmod \text{\CHANNELCOUNT})})}$\;
        $u_{i - 1} \leftarrow \hyperref[wavpack:flush_residual]{\FLUSH(u_{i - 2}~,~\text{\ZEROES}_{i - 1}~,~m_{i - 1}~,~\text{\OFFSET}_{i - 1}~,~\text{\ADD}_{i - 1}~,~\text{\SIGN}_{i - 1}~,~m_i)}$\;
      }
    }
  }(\tcc*[f]{encode regular residual}){
    $\text{\ZEROES}_i \leftarrow \text{\UNDEFINED}$\;
    $(m_i~,~\text{\OFFSET}_i~,~\text{\ADD}_i~,~\text{\SIGN}_i) \leftarrow \hyperref[wavpack:encode_residual]{\ENCODERESIDUAL(r_i~,~\text{\ENTROPY}_{(i \bmod \text{\CHANNELCOUNT})})}$\;
    $u_{i - 1} \leftarrow \hyperref[wavpack:flush_residual]{\FLUSH(u_{i - 2}~,~\text{\ZEROES}_{i - 1}~,~m_{i - 1}~,~\text{\OFFSET}_{i - 1}~,~\text{\ADD}_{i - 1}~,~\text{\SIGN}_{i - 1}~,~m_i)}$\;
  }
  $i \leftarrow i + 1$\;
}
\BlankLine
\tcc{flush final residual}
$u_{i - 1} \leftarrow \hyperref[wavpack:flush_residual]{\FLUSH(u_{i - 2}~,~\text{\ZEROES}_{i - 1}~,~m_{i - 1}~,~\text{\OFFSET}_{i - 1}~,~\text{\ADD}_{i - 1}~,~\text{\SIGN}_{i - 1}~,~0)}$\;
\BlankLine
\Return encoded bitstream sub block\;
\EALGORITHM

\subsubsection{\texttt{unary\_undefined} Function}
\ALGORITHM{$u_{i - 2}$, $m_{i - 1}$}{whether $u_{i - 1}$ is defined}
\SetKw{AND}{and}
\SetKw{IS}{is}
\SetKw{NOT}{not}
\SetKwData{TRUE}{true}
\SetKwData{FALSE}{false}
\SetKwData{UNDEFINED}{undefined}
\uIf{$m_{i - 1} = \text{\UNDEFINED}$}{
  \Return \TRUE\tcc*[r]{$u_{i - 1}$ \IS \UNDEFINED}
}
\uElseIf{$(m_{i - 1} = 0)$ \AND $(u_{i - 2} \neq \UNDEFINED)$ \AND $(u_{i - 2} \bmod 2 = 0)$}{
  \Return \TRUE\tcc*[r]{$u_{i - 1}$ \IS \UNDEFINED}
}
\Else{
  \Return \FALSE\tcc*[r]{$u_{i - 1}$ \IS \NOT \UNDEFINED}
}
\EALGORITHM
}

\clearpage

\subsubsection{\texttt{encode\_residual} Function}
\label{wavpack:encode_residual}
{\relsize{-1}
  \ALGORITHM{signed residual value, entropy variables for channel}{$\text{m}_i$, $\text{offset}_i$, $\text{add}_i$, $\text{sign}_i$; updated entropy values}
  \SetKwData{RESIDUAL}{residual}
  \SetKwData{UNSIGNED}{unsigned}
  \SetKwData{SIGN}{sign}
  \SetKwData{ENTROPY}{entropy}
  \SetKwData{MEDIAN}{median}
  \SetKwData{MED}{m}
  \SetKwData{ADD}{add}
  \SetKwData{OFFSET}{offset}
  \eIf{$\RESIDUAL \geq 0$}{
    $\UNSIGNED \leftarrow \RESIDUAL$\;
    $\text{\SIGN}_i \leftarrow 0$\;
  }{
    $\UNSIGNED \leftarrow -\RESIDUAL - 1$\;
    $\text{\SIGN}_i \leftarrow 1$\;
  }
  $\text{\MEDIAN}_{c~0} \leftarrow \lfloor\text{\ENTROPY}_{c~0} \div 2 ^ 4\rfloor + 1$\;
  $\text{\MEDIAN}_{c~1} \leftarrow \lfloor\text{\ENTROPY}_{c~1} \div 2 ^ 4\rfloor + 1$\;
  $\text{\MEDIAN}_{c~2} \leftarrow \lfloor\text{\ENTROPY}_{c~2} \div 2 ^ 4\rfloor + 1$\;
  \uIf{$\UNSIGNED < \text{\MEDIAN}_{c~0}$}{
    $\text{\MED}_i \leftarrow 0$\;
    $\text{\OFFSET}_i \leftarrow \text{\UNSIGNED}$\tcc*[r]{offset is unsigned - base}
    $\text{\ADD}_i \leftarrow \text{\MEDIAN}_{c~0} - 1$\;
    $\text{\ENTROPY}_{c~0} \leftarrow \text{\ENTROPY}_{c~0} - \lfloor(\text{\ENTROPY}_{c~0} + 126) \div 128\rfloor \times 2$\;
  }
  \uElseIf{$(\UNSIGNED - \text{\MEDIAN}_{c~0}) < \text{\MEDIAN}_{c~1}$}{
    $\text{\MED}_i \leftarrow 1$\;
    $\text{\OFFSET}_i \leftarrow \text{\UNSIGNED} - \text{\MEDIAN}_{c~0}$\;
    $\text{\ADD}_i \leftarrow \text{\MEDIAN}_{c~1} - 1$\;
    $\text{\ENTROPY}_{c~0} \leftarrow \text{\ENTROPY}_{c~0} + \lfloor(\text{\ENTROPY}_{c~0} + 128) \div 128\rfloor \times 5$\;
    $\text{\ENTROPY}_{c~1} \leftarrow \text{\ENTROPY}_{c~1} - \lfloor(\text{\ENTROPY}_{c~1} + 62) \div 64\rfloor \times 2$\;
  }
  \uElseIf{$(\UNSIGNED - (\text{\MEDIAN}_{c~0} + \text{\MEDIAN}_{c~1})) < \text{\MEDIAN}_{c~2}$}{
    $\text{\MED}_i \leftarrow 2$\;
    $\text{\OFFSET}_i \leftarrow \text{\UNSIGNED} - (\text{\MEDIAN}_{c~0} + \text{\ENTROPY}_{c~1})$\;
    $\text{\ADD}_i \leftarrow \text{\MEDIAN}_{c~2} - 1$\;
    $\text{\ENTROPY}_{c~0} \leftarrow \text{\ENTROPY}_{c~0} + \lfloor(\text{\ENTROPY}_{c~0} + 128) \div 128\rfloor \times 5$\;
    $\text{\ENTROPY}_{c~1} \leftarrow \text{\ENTROPY}_{c~1} + \lfloor(\text{\ENTROPY}_{c~1} + 64) \div 64\rfloor \times 5$\;
    $\text{\ENTROPY}_{c~2} \leftarrow \text{\ENTROPY}_{c~2} - \lfloor(\text{\ENTROPY}_{c~2} + 30) \div 32\rfloor \times 2$\;
  }
  \Else{
    $\text{\MED}_i \leftarrow \lfloor(\UNSIGNED - (\text{\MEDIAN}_{c~0} + \text{\MEDIAN}_{c~1})) \div \text{\MEDIAN}_{c~2}\rfloor + 2$\;
    $\text{\OFFSET}_i \leftarrow \text{\UNSIGNED} - (\text{\MEDIAN}_{c~0} + \text{\MEDIAN}_{c~1} + ((\text{\MED}_i - 2) \times \text{\MEDIAN}_{c~2}))$\;
    $\text{\ADD}_i \leftarrow \text{\MEDIAN}_{c~2} - 1$\;
    $\text{\ENTROPY}_{c~0} \leftarrow \text{\ENTROPY}_{c~0} + \lfloor(\text{\ENTROPY}_{c~0} + 128) \div 128\rfloor \times 5$\;
    $\text{\ENTROPY}_{c~1} \leftarrow \text{\ENTROPY}_{c~1} + \lfloor(\text{\ENTROPY}_{c~1} + 64) \div 64\rfloor \times 5$\;
    $\text{\ENTROPY}_{c~2} \leftarrow \text{\ENTROPY}_{c~2} + \lfloor(\text{\ENTROPY}_{c~2} + 32) \div 32\rfloor \times 5$\;
  }
  \Return $\left\lbrace\begin{tabular}{l}
  $\text{\MED}_i$ \\
  $\text{\OFFSET}_i$ \\
  $\text{\ADD}_i$ \\
  $\text{\SIGN}_i$ \\
  $\text{\ENTROPY}_c$ \\
  \end{tabular}\right.$\;
  \EALGORITHM
}

\subsubsection{Writing Modified Elias Gamma Code}
\label{wavpack:write_egc}
\ALGORITHM{a non-negative integer value}{an encoded value}
\eIf{$v \leq 1$}{
  $v \rightarrow$ \WUNARY with stop bit 0\;
}{
  $t \leftarrow \lfloor\log_2(v)\rfloor + 1$\;
  $t \rightarrow$ \WUNARY with stop bit 0\;
  $v \bmod 2 ^ {t - 1} \rightarrow$ \WRITE $(t - 1)$ unsigned bits\;
}
\Return encoded value\;
\EALGORITHM

\clearpage

\subsubsection{\texttt{flush} Function}
\label{wavpack:flush_residual}
{\relsize{-1}
  \ALGORITHM{$u_{i - 2}$, $\text{zeroes}_{i - 1}$, $m_{i - 1}$, $\text{offset}_{i - 1}$, $\text{add}_{i - 1}$, $\text{sign}_{i - 1}$, $m_i$}{$u_{i - 1}$}
  \SetKwData{UNDEFINED}{undefined}
  \SetKwData{ZEROES}{zeroes}
  \SetKwData{OFFSET}{offset}
  \SetKwData{ADD}{add}
  \SetKwData{SIGN}{sign}
  \SetKwFunction{UNARY}{unary}
  \SetKwFunction{WRITERESIDUAL}{write\_residual}
  \SetKw{AND}{and}
  \SetKw{OR}{or}
  \If{$\text{\ZEROES}_{i - 1} \neq \text{\UNDEFINED}$}{
    write $\text{\ZEROES}_{i - 1}$ \hyperref[wavpack:write_egc]{in modified Elias gamma code}\;
  }
  \eIf{$m_{i - 1} \neq \text{\UNDEFINED}$}{
    \tcc{calculate $\text{unary}_{i - 1}$ value for $m_{i - 1}$ based on $m_i$}
    \uIf{$(m_{i - 1} > 0)$ \AND $(m_i > 0)$}{
      \eIf{$(u_{i - 2} = \UNDEFINED)$ \OR $(u_{i - 2} \bmod 2 = 0)$}{
        $u_{i - 1} \leftarrow (m_{i - 1} \times 2) + 1$\;
      }{
        $u_{i - 1} \leftarrow (m_{i - 1} \times 2) - 1$\;
      }
    }
    \uElseIf{$(m_{i - 1} = 0)$ \AND $(m_i > 0)$}{
      \eIf{$(u_{i - 2} = \UNDEFINED)$ \OR $(u_{i - 2} \bmod 2 = 1)$}{
        $u_{i - 1} \leftarrow 1$\;
      }{
        $u_{i - 1} \leftarrow \UNDEFINED$\;
      }
    }
    \uElseIf{$(m_{i - 1} > 0)$ \AND $(m_i = 0)$}{
      \eIf{$(u_{i - 2} = \UNDEFINED)$ \OR $(u_{i - 2} \bmod 2 = 0)$}{
        $u_{i - 1} \leftarrow m_{i - 1} \times 2$\;
      }{
        $u_{i - 1} \leftarrow (m_{i - 1} - 1) \times 2$\;
      }
    }
    \ElseIf{$(m_{i - 1} = 0)$ \AND $(m_i = 0)$}{
      \eIf{$(u_{i - 2} = \UNDEFINED)$ \OR $(u_{i - 2} \bmod 2 = 1)$}{
        $u_{i - 1} \leftarrow 0$\;
      }{
        $u_{i - 1} \leftarrow \UNDEFINED$\;
      }
    }
    \BlankLine
    \tcc{write $\text{residual}_{i - 1}$ to disk using $\text{unary}_{i - 1}$}
    \If{$u_{i - 1} \neq \UNDEFINED$}{
      \eIf{$u_{i - 1} < 16$}{
        $u_{i - 1} \rightarrow$ \WUNARY with stop bit 0\;
      }{
        $16 \rightarrow$ \WUNARY with stop bit 0\;
        $u_{i - 1} - 16 \rightarrow$ \hyperref[wavpack:write_egc]{write in modified Elias gamma code}\;
      }
    }
    \If{$\text{\ADD}_{i - 1} > 0$}{
      $p \leftarrow \lfloor\log_2(\text{\ADD}_{i - 1})\rfloor$\;
      $e \leftarrow 2 ^ {p + 1} - \text{\ADD}_{i - 1} - 1$\;
      \eIf{$\text{\OFFSET}_{i - 1} < e$}{
        $r \leftarrow \text{\OFFSET}_{i - 1}$\;
        $r \rightarrow$ \WRITE $p$ unsigned bits\;
      }{
        $r \leftarrow \lfloor(\text{\OFFSET}_{i - 1} + e) \div 2\rfloor$\;
        $b \leftarrow (\text{\OFFSET}_{i - 1} + e) \bmod 2$\;
        $r \rightarrow$ \WRITE $p$ unsigned bits\;
        $b \rightarrow$ \WRITE 1 unsigned bit\;
      }
    }
    $\text{\SIGN}_{i - 1} \rightarrow$ \WRITE 1 unsigned bit\;
  }{
    $u_{i - 1} \leftarrow \text{\UNDEFINED}$\;
  }
  \Return $u_{i - 1}$\;
  \EALGORITHM
}

\clearpage


\begin{landscape}

\subsubsection{Residual Encoding Example}
{\relsize{-2}
\renewcommand{\arraystretch}{1.75}
\begin{tabular}{|>{$}r<{$}||>{$}r<{$}|>{$}r<{$}|>{$}r<{$}||>{$}r<{$}|>{$}r<{$}|>{$}r<{$}||>{$}r<{$}|>{$}r<{$}|>{$}r<{$}|}
i & r_i & \text{unsigned}_i &\text{sign}_i & \text{median}_{c~0} & \text{median}_{c~1} & \text{median}_{c~2} & m_i & \text{offset}_i & \text{add}_i \\
\hline
0 & -61 &
60 & 1 &
\left\lfloor\frac{118}{2 ^ 4}\right\rfloor + 1 = 8 & \left\lfloor\frac{194}{2 ^ 4}\right\rfloor + 1 = 13 & \left\lfloor\frac{322}{2 ^ 4}\right\rfloor + 1 = 21 &
3 & 60 - (8 + 13 + ((3 - 2) \times 21)) = 18 & 21 - 1 = 20
\\
1 & 31 &
31 & 0 &
\left\lfloor\frac{118}{2 ^ 4}\right\rfloor + 1 = 8 & \left\lfloor\frac{176}{2 ^ 4}\right\rfloor + 1 = 12 & \left\lfloor\frac{212}{2 ^ 4}\right\rfloor + 1 = 14 &
2 & 31 - (8 + 12) = 11 & 14 - 1 = 13
\\
\hline
2 & -33 &
32 & 1 &
\left\lfloor\frac{123}{2 ^ 4}\right\rfloor + 1 = 8 & \left\lfloor\frac{214}{2 ^ 4}\right\rfloor + 1 = 14 & \left\lfloor\frac{377}{2 ^ 4}\right\rfloor + 1 = 24 &
2 & 32 - (8 + 14) = 10 & 24 - 1 = 23
\\
3 & 32 &
32 & 0 &
\left\lfloor\frac{123}{2 ^ 4}\right\rfloor + 1 = 8 & \left\lfloor\frac{191}{2 ^ 4}\right\rfloor + 1 = 12 & \left\lfloor\frac{198}{2 ^ 4}\right\rfloor + 1 = 13 &
2 & 32 - (8 + 12) = 12 & 13 - 1 = 12
\\
\hline
4 & -18 &
17 & 1 &
\left\lfloor\frac{128}{2 ^ 4}\right\rfloor + 1 = 9 & \left\lfloor\frac{234}{2 ^ 4}\right\rfloor + 1 = 15 & \left\lfloor\frac{353}{2 ^ 4}\right\rfloor + 1 = 23 &
1 & 17 - 9 = 8 & 15 - 1 = 14
\\
5 & 36 &
36 & 0 &
\left\lfloor\frac{128}{2 ^ 4}\right\rfloor + 1 = 9 & \left\lfloor\frac{206}{2 ^ 4}\right\rfloor + 1 = 13 & \left\lfloor\frac{184}{2 ^ 4}\right\rfloor + 1 = 12 &
3 & 36 - (9 + 13 + ((3 - 2) \times 12)) = 2 & 12 - 1 = 11
\\
\hline
6 & 1 &
1 & 0 &
\left\lfloor\frac{138}{2 ^ 4}\right\rfloor + 1 = 9 & \left\lfloor\frac{226}{2 ^ 4}\right\rfloor + 1 = 15 & \left\lfloor\frac{353}{2 ^ 4}\right\rfloor + 1 = 23 &
0 & 1 & 9 - 1 = 8
\\
7 & 37 &
37 & 0 &
\left\lfloor\frac{138}{2 ^ 4}\right\rfloor + 1 = 9 & \left\lfloor\frac{226}{2 ^ 4}\right\rfloor + 1 = 15 & \left\lfloor\frac{214}{2 ^ 4}\right\rfloor + 1 = 14 &
2 & 37 - (9 + 15) = 13 & 14 - 1 = 13
\\
\hline
8 & 20 &
20 & 0 &
\left\lfloor\frac{134}{2 ^ 4}\right\rfloor + 1 = 9 & \left\lfloor\frac{226}{2 ^ 4}\right\rfloor + 1 = 15 & \left\lfloor\frac{353}{2 ^ 4}\right\rfloor + 1 = 23 &
1 & 20 - 9 = 11 & 15 - 1 = 14
\\
9 & 35 &
35 & 0 &
\left\lfloor\frac{148}{2 ^ 4}\right\rfloor + 1 = 10 & \left\lfloor\frac{246}{2 ^ 4}\right\rfloor + 1 = 16 & \left\lfloor\frac{200}{2 ^ 4}\right\rfloor + 1 = 13 &
2 & 35 - (10 + 16) = 9 & 13 - 1 = 12
\\
\hline
10 & 35 &
35 & 0 &
\left\lfloor\frac{144}{2 ^ 4}\right\rfloor + 1 = 10 & \left\lfloor\frac{218}{2 ^ 4}\right\rfloor + 1 = 14 & \left\lfloor\frac{353}{2 ^ 4}\right\rfloor + 1 = 23 &
2 & 35 - (10 + 14) = 11 & 23 - 1 = 22
\\
11 & 31 &
31 & 0 &
\left\lfloor\frac{158}{2 ^ 4}\right\rfloor + 1 = 10 & \left\lfloor\frac{266}{2 ^ 4}\right\rfloor + 1 = 17 & \left\lfloor\frac{186}{2 ^ 4}\right\rfloor + 1 = 12 &
2 & 31 - (10 + 17) = 4 & 12 - 1 = 11
\\
\hline
12 & 50 &
50 & 0 &
\left\lfloor\frac{154}{2 ^ 4}\right\rfloor + 1 = 10 & \left\lfloor\frac{238}{2 ^ 4}\right\rfloor + 1 = 15 & \left\lfloor\frac{331}{2 ^ 4}\right\rfloor + 1 = 21 &
3 & 50 - (10 + 15 + ((3 - 2) \times 21)) = 4 & 21 - 1 = 20
\\
13 & 25 &
25 & 0 &
\left\lfloor\frac{168}{2 ^ 4}\right\rfloor + 1 = 11 & \left\lfloor\frac{291}{2 ^ 4}\right\rfloor + 1 = 19 & \left\lfloor\frac{174}{2 ^ 4}\right\rfloor + 1 = 11 &
1 & 25 - 11 = 14 & 19 - 1 = 18
\\
\hline
14 & 62 &
62 & 0 &
\left\lfloor\frac{164}{2 ^ 4}\right\rfloor + 1 = 11 & \left\lfloor\frac{258}{2 ^ 4}\right\rfloor + 1 = 17 & \left\lfloor\frac{386}{2 ^ 4}\right\rfloor + 1 = 25 &
3 & 62 - (11 + 17 + ((3 - 2) \times 25)) = 9 & 25 - 1 = 24
\\
15 & 18 &
18 & 0 &
\left\lfloor\frac{178}{2 ^ 4}\right\rfloor + 1 = 12 & \left\lfloor\frac{281}{2 ^ 4}\right\rfloor + 1 = 18 & \left\lfloor\frac{174}{2 ^ 4}\right\rfloor + 1 = 11 &
1 & 18 - 12 = 6 & 18 - 1 = 17
\\
\hline
16 & 68 &
68 & 0 &
\left\lfloor\frac{174}{2 ^ 4}\right\rfloor + 1 = 11 & \left\lfloor\frac{283}{2 ^ 4}\right\rfloor + 1 = 18 & \left\lfloor\frac{451}{2 ^ 4}\right\rfloor + 1 = 29 &
3 & 68 - (11 + 18 + ((3 - 2) \times 29)) = 10 & 29 - 1 = 28
\\
17 & 10 &
10 & 0 &
\left\lfloor\frac{188}{2 ^ 4}\right\rfloor + 1 = 12 & \left\lfloor\frac{271}{2 ^ 4}\right\rfloor + 1 = 17 & \left\lfloor\frac{174}{2 ^ 4}\right\rfloor + 1 = 11 &
0 & 10 & 12 - 1 = 11
\\
\hline
18 & 71 &
71 & 0 &
\left\lfloor\frac{184}{2 ^ 4}\right\rfloor + 1 = 12 & \left\lfloor\frac{308}{2 ^ 4}\right\rfloor + 1 = 20 & \left\lfloor\frac{526}{2 ^ 4}\right\rfloor + 1 = 33 &
3 & 71 - (12 + 20 + ((3 - 2) \times 33)) = 6 & 33 - 1 = 32
\\
19 & 0 &
0 & 0 &
\left\lfloor\frac{184}{2 ^ 4}\right\rfloor + 1 = 12 & \left\lfloor\frac{271}{2 ^ 4}\right\rfloor + 1 = 17 & \left\lfloor\frac{174}{2 ^ 4}\right\rfloor + 1 = 11 &
0 & 0 & 12 - 1 = 11
\\
\hline
\end{tabular}
\renewcommand{\arraystretch}{1.0}
}

\clearpage

{\relsize{-2}
\renewcommand{\arraystretch}{1.75}
\begin{tabular}{|>{$}r<{$}||>{$}r<{$}|>{$}r<{$}|>{$}r<{$}||>{$}r<{$}|>{$}r<{$}|>{$}r<{$}|>{$}r<{$}|>{$}r<{$}|}
i & m_i & \text{offset}_i & \text{add}_i & u_i & p_i & e_i & r_i & b_i \\
\hline
0 & 3 &
18 & 20 &
(3 \times 2) + 1 = 7 &
\lfloor\log_2(20)\rfloor = 4 &
2 ^ {(4 + 1)} - 20 - 1 = 11 &
\lfloor(18 + 11) \div 2 \rfloor = 14 &
(18 + 11) \bmod 2  = 1 \\
1 & 2 &
11 & 13 &
(2 \times 2) - 1 = 3 &
\lfloor\log_2(13)\rfloor = 3 &
2 ^ {(3 + 1)} - 13 - 1 = 2 &
\lfloor(11 + 2) \div 2 \rfloor = 6 &
(11 + 2) \bmod 2  = 1 \\
\hline
2 & 2 &
10 & 23 &
(2 \times 2) - 1 = 3 &
\lfloor\log_2(23)\rfloor = 4 &
2 ^ {(4 + 1)} - 23 - 1 = 8 &
\lfloor(10 + 8) \div 2 \rfloor = 9 &
(10 + 8) \bmod 2  = 0 \\
3 & 2 &
12 & 12 &
(2 \times 2) - 1 = 3 &
\lfloor\log_2(12)\rfloor = 3 &
2 ^ {(3 + 1)} - 12 - 1 = 3 &
\lfloor(12 + 3) \div 2 \rfloor = 7 &
(12 + 3) \bmod 2  = 1 \\
\hline
4 & 1 &
8 & 14 &
(1 \times 2) - 1 = 1 &
\lfloor\log_2(14)\rfloor = 3 &
2 ^ {(3 + 1)} - 14 - 1 = 1 &
\lfloor(8 + 1) \div 2 \rfloor = 4 &
(8 + 1) \bmod 2  = 1 \\
5 & 3 &
2 & 11 &
(3 - 1) \times 2 = 4 &
\lfloor\log_2(11)\rfloor = 3 &
2 ^ {(3 + 1)} - 11 - 1 = 4 &
2 & \\
\hline
6 & 0 &
1 & 8 &
\textit{undefined} &
\lfloor\log_2(8)\rfloor = 3 &
2 ^ {(3 + 1)} - 8 - 1 = 7 &
1 & \\
7 & 2 &
13 & 13 &
(2 \times 2) + 1 = 5 &
\lfloor\log_2(13)\rfloor = 3 &
2 ^ {(3 + 1)} - 13 - 1 = 2 &
\lfloor(13 + 2) \div 2 \rfloor = 7 &
(13 + 2) \bmod 2  = 1 \\
\hline
8 & 1 &
11 & 14 &
(1 \times 2) - 1 = 1 &
\lfloor\log_2(14)\rfloor = 3 &
2 ^ {(3 + 1)} - 14 - 1 = 1 &
\lfloor(11 + 1) \div 2 \rfloor = 6 &
(11 + 1) \bmod 2  = 0 \\
9 & 2 &
9 & 12 &
(2 \times 2) - 1 = 3 &
\lfloor\log_2(12)\rfloor = 3 &
2 ^ {(3 + 1)} - 12 - 1 = 3 &
\lfloor(9 + 3) \div 2 \rfloor = 6 &
(9 + 3) \bmod 2  = 0 \\
\hline
10 & 2 &
11 & 22 &
(2 \times 2) - 1 = 3 &
\lfloor\log_2(22)\rfloor = 4 &
2 ^ {(4 + 1)} - 22 - 1 = 9 &
\lfloor(11 + 9) \div 2 \rfloor = 10 &
(11 + 9) \bmod 2  = 0 \\
11 & 2 &
4 & 11 &
(2 \times 2) - 1 = 3 &
\lfloor\log_2(11)\rfloor = 3 &
2 ^ {(3 + 1)} - 11 - 1 = 4 &
\lfloor(4 + 4) \div 2 \rfloor = 4 &
(4 + 4) \bmod 2  = 0 \\
\hline
12 & 3 &
4 & 20 &
(3 \times 2) - 1 = 5 &
\lfloor\log_2(20)\rfloor = 4 &
2 ^ {(4 + 1)} - 20 - 1 = 11 &
4 & \\
13 & 1 &
14 & 18 &
(1 \times 2) - 1 = 1 &
\lfloor\log_2(18)\rfloor = 4 &
2 ^ {(4 + 1)} - 18 - 1 = 13 &
\lfloor(14 + 13) \div 2 \rfloor = 13 &
(14 + 13) \bmod 2  = 1 \\
\hline
14 & 3 &
9 & 24 &
(3 \times 2) - 1 = 5 &
\lfloor\log_2(24)\rfloor = 4 &
2 ^ {(4 + 1)} - 24 - 1 = 7 &
\lfloor(9 + 7) \div 2 \rfloor = 8 &
(9 + 7) \bmod 2  = 0 \\
15 & 1 &
6 & 17 &
(1 \times 2) - 1 = 1 &
\lfloor\log_2(17)\rfloor = 4 &
2 ^ {(4 + 1)} - 17 - 1 = 14 &
6 & \\
\hline
16 & 3 &
10 & 28 &
(3 - 1) \times 2 = 4 &
\lfloor\log_2(28)\rfloor = 4 &
2 ^ {(4 + 1)} - 28 - 1 = 3 &
\lfloor(9 + 3) \div 2 \rfloor = 6 &
(9 + 3) \bmod 2  = 0 \\
17 & 0 &
10 & 11 &
\textit{undefined} &
\lfloor\log_2(11)\rfloor = 3 &
2 ^ {(3 + 1)} - 11 - 1 = 4 &
\lfloor(10 + 4) \div 2 \rfloor = 7 &
(10 + 4) \bmod 2  = 0 \\
\hline
18 & 3 &
6 & 32 &
3 \times 2 = 6 &
\lfloor\log_2(32)\rfloor = 5 &
2 ^ {(5 + 1)} - 32 - 1 = 31 &
6 & \\
19 & 0 &
0 & 11 &
\textit{undefined} &
\lfloor\log_2(11)\rfloor = 3 &
2 ^ {(3 + 1)} - 11 - 1 = 4 &
0 & \\
\hline
\end{tabular}
\renewcommand{\arraystretch}{1.0}
}

\end{landscape}

\subsubsection{2nd Residual Encoding Example}
{\relsize{-2}
\renewcommand{\arraystretch}{1.75}
\begin{tabular}{|r|r|>{$}r<{$}>{$}r<{$}>{$}r<{$}||>{$}r<{$}|>{$}r<{$}>{$}r<{$}>{$}r<{$}>{$}r<{$}>{$}r<{$}|l}
  $i$ & $r_i$ & \text{entropy}_{0~0} & \text{entropy}_{0~1} & \text{entropy}_{0~2}  & u_i & \text{zeroes}_i & m_i & \text{offset}_i & \text{add}_i & \text{sign}_i \\
\cline{0-10}
-2 & & & & & \textit{und.} & & & & & \\
-1 & & {\color{red}0} & 0 & 0 & {\color{red}\textit{und.}} & \textit{und.} & \textit{und.} & \textit{und.} & \textit{und.} & \textit{und.} \\
0 & 1 & 0 & 0 & 0 & 3 & {\color{blue}0} & 1 & 0 & 0 & 0 \\
1 & 2 & 5 & 0 & 0 & 3 & \textit{und.} & 2 & 0 & 0 & 0 \\
2 & 3 & 10 & 5 & 0 & 5 & \textit{und.} & 3 & 0 & 0 & 0 \\
3 & 2 & 15 & 10 & 5 & 2 & \textit{und.} & 2 & 0 & 0 & 0 \\
4 & 1 & 20 & 15 & 3 & \textit{und.} & \textit{und.} & 0 & 1 & 1 & 0 \\
5 & 0 & 18 & 15 & 3 & 0 & \textit{und.} & 0 & 0 & 1 & 0 \\
6 & 0 & 16 & 15 & 3 & \textit{und.} & \textit{und.} & 0 & 0 & 1 & 0 \\
7 & 0 & 14 & 15 & 3 & 0 & \textit{und.} & 0 & 0 & 0 & 0 \\
8 & 0 & 12 & 15 & 3 & \textit{und.} & \textit{und.} & 0 & 0 & 0 & 0 \\
9 & 0 & 10 & 15 & 3 & 0 & \textit{und.} & 0 & 0 & 0 & 0 \\
10 & 0 & 8 & 15 & 3 & \textit{und.} & \textit{und.} & 0 & 0 & 0 & 0 \\
11 & 0 & 6 & 15 & 3 & 0 & \textit{und.} & 0 & 0 & 0 & 0 \\
12 & 0 & 4 & 15 & 3 & \textit{und.} & \textit{und.} & 0 & 0 & 0 & 0 \\
13 & 0 & 2 & 15 & 3 & 0 & \textit{und.} & 0 & 0 & 0 & 0 \\
14 & 0 & {\color{red}0} & 15 & 3 & {\color{red}\textit{und.}} & \textit{und.} & 0 & 0 & 0 & 0 \\
\cline{0-10}
15 & 0 & 0 & 15 & 3 & \textit{und.} & 1 & \textit{und.} & \textit{und.} & \textit{und.} & \textit{und.} & \multirow{10}{1em}{\begin{sideways}block of 10 zero residuals\end{sideways}} \\
16 & 0 & 0 & 0 & 0 & \textit{und.} & 2 & \textit{und.} & \textit{und.} & \textit{und.} & \textit{und.} \\
17 & 0 & 0 & 0 & 0 & \textit{und.} & 3 & \textit{und.} & \textit{und.} & \textit{und.} & \textit{und.} \\
18 & 0 & 0 & 0 & 0 & \textit{und.} & 4 & \textit{und.} & \textit{und.} & \textit{und.} & \textit{und.} \\
19 & 0 & 0 & 0 & 0 & \textit{und.} & 5 & \textit{und.} & \textit{und.} & \textit{und.} & \textit{und.} \\
20 & 0 & 0 & 0 & 0 & \textit{und.} & 6 & \textit{und.} & \textit{und.} & \textit{und.} & \textit{und.} \\
21 & 0 & 0 & 0 & 0 & \textit{und.} & 7 & \textit{und.} & \textit{und.} & \textit{und.} & \textit{und.} \\
22 & 0 & 0 & 0 & 0 & \textit{und.} & 8 & \textit{und.} & \textit{und.} & \textit{und.} & \textit{und.} \\
23 & 0 & 0 & 0 & 0 & \textit{und.} & 9 & \textit{und.} & \textit{und.} & \textit{und.} & \textit{und.} \\
24 & 0 & 0 & 0 & 0 & \textit{und.} & 10 & \textit{und.} & \textit{und.} & \textit{und.} & \textit{und.} \\
\cline{0-10}
25 & -1 & 0 & 0 & 0 & 1 & {\color{blue}10} & 0 & 0 & 0 & 1 \\
26 & -2 & 0 & 0 & 0 & 1 & \textit{und.} & 1 & 0 & 0 & 1 \\
27 & -3 & 5 & 0 & 0 & 3 & \textit{und.} & 2 & 0 & 0 & 1 \\
28 & -2 & 10 & 5 & 0 & 0 & \textit{und.} & 1 & 0 & 0 & 1 \\
29 & -1 & 15 & 3 & 0 & \textit{und.} & \textit{und.} & 0 & 0 & 0 & 1 \\
\cline{0-10}
\end{tabular}
}

\clearpage

This example is more simplified to demonstrate how the \VAR{zeroes}
value propagates in two instances.

For $r_0$, because $\text{entropy}_{0~0} = 0$ and
$u_{(-1)} = \textit{undefined}$\footnote{as determined by the \texttt{unary\_undefined} function},
we must handle a block of zeroes in some way.
But because $r_0 \neq 0$, we prepend a ``false alarm'' block of zeroes
and encode the residual normally.

For $r_{15}$, because $\text{entropy}_{0~0} = 0$,
$u_{14} = \textit{undefined}$ and $r_{15} = 0$,
we flush $\text{residual}_{14}$'s values and begin a block of zeroes
which continues until $r_{25} \neq 0$.
This block of zeroes is prepended to $\text{residual}_{25}$'s
values, which are flushed once $\text{residual}_{26}$ is encoded
and $u_{25}$ can be calculated from $u_{24}$ and $m_{26}$.

\begin{figure}[h]
  \includegraphics{wavpack/figures/residuals_parse2.pdf}
  \caption{2nd residual encoding example}
\end{figure}

\begin{figure}[h]
  \includegraphics{wavpack/figures/residuals.pdf}
\end{figure}


\clearpage

\subsection{Writing RIFF WAVE Header and Footer}
\label{wavpack:write_wave_header}
\begin{figure}[h]
  \includegraphics{wavpack/figures/pcm_sandwich.pdf}
\end{figure}


\subsection{Writing MD5 Sum}
\label{wavpack:write_md5}
MD5 sum is calculated as if the PCM data had been read from
a wave file's \texttt{data} chunk.
That is, the samples are converted to little-endian format
and are signed if the stream's bits-per-sample is greater than 8.
\begin{figure}[h]
  \includegraphics{wavpack/figures/md5sum.pdf}
\end{figure}

\subsection{Writing Extended Integers}
\label{wavpack:write_extended_integers}
\begin{figure}[h]
  \includegraphics{wavpack/figures/extended_integers.pdf}
\end{figure}

\clearpage

\subsection{Writing Channel Info}
\label{wavpack:write_channel_info}
\begin{figure}[h]
  \includegraphics{wavpack/figures/channel_info.pdf}
\end{figure}


\subsection{Writing Sample Rate}
\label{wavpack:write_sample_rate}
\begin{figure}[h]
  \includegraphics{wavpack/figures/sample_rate.pdf}
\end{figure}

\clearpage

\subsection{Writing Block Header}
\label{wavpack:write_block_header}
{\relsize{-1}
  \input{wavpack/algorithms/write_block_header}
}

\clearpage

\subsection{Round-Tripping Correlation Weights}
\label{wavpack:roundtrip_weights}
Because the final weight values of one block
may not be exactly representable in a correlation weights sub-block,
it's necessary to ``round-trip'' the weight values
so that the starting values for the next block
are the same as the values stored in the sub-block.

\input{wavpack/algorithms/roundtrip_weights}

\clearpage

\subsection{Round-Tripping Correlation Samples}
\label{wavpack:roundtrip_samples}

\input{wavpack/algorithms/roundtrip_samples}
