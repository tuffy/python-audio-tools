%This work is licensed under the
%Creative Commons Attribution-Share Alike 3.0 United States License.
%To view a copy of this license, visit
%http://creativecommons.org/licenses/by-sa/3.0/us/ or send a letter to
%Creative Commons,
%171 Second Street, Suite 300,
%San Francisco, California, 94105, USA.

\chapter{WavPack}
WavPack is a format for compressing Wave files, typically in lossless mode.
Notably, it also has a lossy mode and even a hybrid mode which allows
the `correction' file to be separated from a lossy core.

Metadata is stored as an APEv2 tag, which is described on page \pageref{apev2}.

Its stream of data is stored little-endian, as described on page
\pageref{bitstreams}.

\section{the WavPack file stream}
\begin{figure}[h]
\includegraphics{figures/wavpack_stream.pdf}
\end{figure}

\pagebreak

\section{the WavPack block header}
\begin{figure}[h]
\includegraphics{figures/wavpack_block_header.pdf}
\end{figure}
\begin{wrapfigure}[10]{r}{1.5in}
\begin{tabular}{|c|r|}
\hline
value & sample rate \\
\hline
\texttt{0000} & 6000 \\
\texttt{0001} & 8000 \\
\texttt{0010} & 9600 \\
\texttt{0011} & 11025 \\
\texttt{0100} & 12000 \\
\texttt{0101} & 16000 \\
\texttt{0110} & 22050 \\
\texttt{0111} & 24000 \\
\texttt{1000} & 32000 \\
\texttt{1001} & 44100 \\
\texttt{1010} & 48000 \\
\texttt{1011} & 64000 \\
\texttt{1100} & 88200 \\
\texttt{1101} & 96000 \\
\texttt{1110} & 192000 \\
\texttt{1111} & reserved \\
\hline
\end{tabular}
\end{wrapfigure}

\VAR{Block Size} is the length of everything in the block past
the \VAR{Block Size} field itself -
or everything in the block past the CRC, minus 24 bytes.

\VAR{Bits per Sample} is one of 4 values:

\begin{inparaenum}
\item[\texttt{00} = ] 8 bps,
\item[\texttt{01} = ] 16 bps,
\item[\texttt{10} = ] 24 bps,
\item[\texttt{11} = ] 32 bps
\end{inparaenum}
.

\VAR{Mono Output} bit indicates the channel count.
If 1, this block has 1 channel.
If 0, this block has 2 channels.
For an audio stream with more than 2 channels,
check the \VAR{Initial Block} and \VAR{Final Block} bits to indicate
the start and end of the channels.  As an example:

\begin{tabular}{c|c|c|c}
Initial Block & Final Block & Mono Output & Channels \\
\hline
1 & 0 & 0 & 2 \\
0 & 0 & 1 & 1 \\
0 & 0 & 1 & 1 \\
0 & 1 & 0 & 2 \\
\hline
\multicolumn{3}{r|}{Total} & 6
\end{tabular}

\clearpage

\subsection{WavPack sub-block header}
\begin{figure}[h]
\includegraphics{figures/wavpack_subblock_header.pdf}
\end{figure}
\par
\noindent
If the \VAR{Large Block} field is 0, the \VAR{Block Size} field is 8 bits long.
If it is 1, the \VAR{Block Size} field is 24 bits long.
The \VAR{Block Size} field is the length of \VAR{Block Data}, in 16-bit
words rather than bytes.
If \VAR{Actual Size 1 Less} is set, that means \VAR{Block Data} doesn't contain
an even number of bytes; it is padded with a single null byte at the
end in order to fit.
If \VAR{Nondecoder Data} is set, that means the decoder does not have
to understand the contents of this particular sub-block in
order to decode the audio.
