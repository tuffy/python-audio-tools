%This work is licensed under the
%Creative Commons Attribution-Share Alike 3.0 United States License.
%To view a copy of this license, visit
%http://creativecommons.org/licenses/by-sa/3.0/us/ or send a letter to
%Creative Commons,
%171 Second Street, Suite 300,
%San Francisco, California, 94105, USA.

\chapter{M4A}
\label{m4a}
M4A is typically AAC audio in a QuickTime container stream, though
it may also contain other formats such as MPEG-1 audio.

\section{the QuickTime File Stream}
\begin{figure}[h]
\includegraphics{figures/m4a/quicktime.pdf}
\end{figure}
\par
\noindent
Unlike other chunked formats such as RIFF WAVE, QuickTime's atom chunks
may be containers for other atoms.  All of its fields are big-endian.
\subsection{a QuickTime Atom}
\begin{figure}[h]
\includegraphics{figures/m4a/atom.pdf}
\end{figure}
\VAR{Atom Type} is an ASCII string.
\VAR{Atom Length} is the length of the entire atom, including the header.
If \VAR{Atom Length} is 0, the atom continues until the end of the file.
If \VAR{Atom Length} is 1, the atom has an extended size.  This means
there is a 64-bit length field immediately after the header which is
the atom's actual size.
\begin{figure}[h]
\includegraphics{figures/m4a/atom2.pdf}
\end{figure}
\subsection{Container Atoms}
There is no flag or field to tell a QuickTime parser which
of its atoms are containers and which ones are not.
If an atom is known to be a container, one can treat its Atom Data
as a QuickTime stream and parse it in a recursive fashion.

\clearpage

\section{M4A Atoms}
A typical M4A begins with an `ftyp' atom indicating its file type,
followed by a `moov' atom containing a copious amount of file metadata,
an optional `free' atom with nothing but empty space
(so that metadata can be resized, if necessary) and an `mdat' atom
containing the song data itself.
\begin{figure}[h]
\parpic[r]{
\includegraphics{figures/m4a/atoms.pdf}
}
\subsection{the ftyp Atom}
\includegraphics{figures/m4a/ftyp.pdf}
\par
The \VAR{Major Brand} and \VAR{Compatible Brand} fields are ASCII strings.
\VAR{Major Brand Version} is an integer.

\subsection{the mvhd Atom}
\includegraphics{figures/m4a/mvhd.pdf}
\par
If \VAR{Version} is 0, \VAR{Created Mac UTC Date}, \VAR{Modified Mac UTC Date} and
\VAR{Duration} are 32-bit fields.  If it is 1, they are 64-bit fields.
\end{figure}

\clearpage

\subsection{the tkhd Atom}
\par
\begin{figure}[h]
\includegraphics{figures/m4a/tkhd.pdf}
\end{figure}
\par
\noindent
As with `mvhd', if \VAR{Version} is 0, \VAR{Created Mac UTC Date},
\VAR{Modified Mac UTC Date} and \VAR{Duration} are 32-bit fields.
If it is 1, they are 64-bit fields.

\subsection{the mdhd Atom}

The \ATOM{mdhd} atom contains track information such as samples-per-second,
track length and creation/modification times.

\begin{figure}[h]
\includegraphics{figures/m4a/mdhd.pdf}
\end{figure}
\par
\noindent
As with `mvhd', if \VAR{Version} is 0, \VAR{Created Mac UTC Date},
\VAR{Modified Mac UTC Date} and \VAR{Track Length} are 32-bit fields.
If it is 1, they are 64-bit fields.

\clearpage

\subsection{the hdlr Atom}

\begin{figure}[h]
\includegraphics{figures/m4a/hdlr.pdf}
\end{figure}
\par
\noindent
\VAR{QuickTime flags}, \VAR{QuickTime flags mask} and \VAR{Component Name Length}
are integers.  The rest are ASCII strings.

\subsection{the smhd Atom}

\begin{figure}[h]
\includegraphics{figures/m4a/smhd.pdf}
\end{figure}

\subsection{the dref Atom}

\begin{figure}[h]
\includegraphics{figures/m4a/dref.pdf}
\end{figure}

\clearpage

\subsection{the stsd Atom}

\begin{figure}[h]
\includegraphics{figures/m4a/stsd.pdf}
\end{figure}

\subsection{the mp4a Atom}

The \ATOM{mp4a} atom contains information such as the number of channels
and bits-per-sample.  It can be found in the \ATOM{stsd} atom.

\begin{figure}[h]
\includegraphics{figures/m4a/mp4a.pdf}
\end{figure}

\clearpage

\subsection{the stts Atom}

\begin{figure}[h]
\includegraphics{figures/m4a/stts.pdf}
\end{figure}

\subsection{the stsc Atom}

\begin{figure}[h]
\includegraphics{figures/m4a/stsc.pdf}
\end{figure}

\subsection{the stsz Atom}

\begin{figure}[h]
\includegraphics{figures/m4a/stsz.pdf}
\end{figure}

\clearpage

\subsection{the stco Atom}

\begin{figure}[h]
\includegraphics{figures/m4a/stco.pdf}
\end{figure}
\par
\noindent
Offsets point to an absolute position in the M4A file of AAC data in
the \ATOM{mdat} atom.  Therefore, if the \ATOM{moov} atom size changes
(which can happen by writing new metadata in its \ATOM{meta} child atom)
the \ATOM{mdat} atom may move and these absolute offsets will change.
In that instance, they \textbf{must}
be re-adjusted in the \ATOM{stco} atom or the file may become unplayable.

\subsection{the meta Atom}
\label{m4a_meta}
\begin{figure}[h]
\parpic[r]{
\includegraphics{figures/m4a/meta_atoms.pdf}
}
\includegraphics{figures/m4a/meta.pdf}
The atoms within the \ATOM{ilst} container are all containers themselves,
each with a \ATOM{data} atom of its own.
Notice that many of \ATOM{ilst}'s sub-atoms begin with the
non-ASCII 0xA9 byte.

\includegraphics{figures/m4a/data.pdf}
\par
\noindent
Text data atoms have a \VAR{Type} of 1.
Binary data atoms typically have a \VAR{Type} of 0.
\end{figure}
\begin{table}[h]
{\relsize{-1}
\begin{tabular}{|r|l||r|l||r|l|}
\hline
Atom & Description & Atom & Description & Atom & Description \\
\hline
\ATOM{alb} & Album Name &
\ATOM{ART} & Track Artist &
\ATOM{cmt} & Comments \\
\ATOM{covr} & Cover Image &
\ATOM{cpil} & Compilation &
\ATOM{cprt} & Copyright \\
\ATOM{day} & Year &
\ATOM{disk} & Disc Number &
\ATOM{gnre} & Genre \\
\ATOM{grp} & Grouping &
\ATOM{----} & iTunes-specific &
\ATOM{nam} & Track Name \\
\ATOM{rtng} & Rating &
\ATOM{tmpo} & BMP &
\ATOM{too} & Encoder \\
\ATOM{trkn} & Track Number &
\ATOM{wrt} & Composer &
& \\
\hline
\end{tabular}
\caption{Known \ATOM{ilst} sub-atoms}
}
\end{table}

\clearpage

\subsubsection{the trkn Sub-Atom}
\ATOM{trkn} is a binary sub-atom of \ATOM{meta} which contains
the track number.
\begin{figure}[h]
\includegraphics{figures/m4a/trkn.pdf}
\end{figure}

\subsubsection{the disk Sub-Atom}
\ATOM{disk} is a binary sub-atom of \ATOM{meta} which contains
the disc number.
For example, if the track belongs to the first disc in a set of
two discs, the sub-atom will contain that information.
\begin{figure}[h]
\includegraphics{figures/m4a/disk.pdf}
\end{figure}
